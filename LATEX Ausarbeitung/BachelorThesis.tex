\documentclass[12pt,a4paper,accentcolor=tud1c,colorback=tud1c,colorbacksubtitle]{tudreport}

%%%%%%%%%%%%%%%%%%%%%%%%%%% Es empfiehlt sich, weitere eigene Befehle vor dem Dokument zu definieren, 
%%%%% Eigene Befehle %%%%%% da sie erst ab der Zeile ihrer Definition verwendet werden können.
%%%%%%%%%%%%%%%%%%%%%%%%%%%

\newcommand{\gf}[1]{\grqq #1\grqq}        %% Befehl \gf{Text} für "Text"

\newcommand{\name}{Elvir Sinancevic} %% Befehl \name fügt den Namen in Klammern{} ein



%%%%%%%%%%%%%%%%%%%%%%%%%%% Zu Beginn müssen alle verwendeten Pakete geladen werden.
%%%%%%%%% Pakete %%%%%%%%%% 
%%%%%%%%%%%%%%%%%%%%%%%%%%%

%%%% Layout & Struktur %%%%

\usepackage[ngerman]{babel} 	% Deutsche Trennregeln

\usepackage[T1]{fontenc}   		% für europäische Autoren ratsam; % wichtig für Trennung von Wörtern mit Umlauten

\usepackage[utf8]{inputenc} 	% Umlaute erkennen (+ ä ö ü)

\usepackage{setspace}			% Zeilenabstand fließend einstellen  
\setstretch{1.0}				% Zeilenabstand festlegen

\usepackage{geometry}			% Neue Seitenlayouts definieren

\usepackage[normalem]{ulem}		% Hilft beim Unterstreichen von Text

\usepackage{xcolor}				% Ermöglicht gefärbten Text und Boxen


%%% Grafiken % Tabellen %%%

\usepackage{graphicx}			% Grafiken einfügen und bearbeiten

\usepackage{wrapfig}			% Grafiken in fließtexte einfügen

\usepackage{multirow} 			% Tabellen mit multiplen Spalten und Zeilen

\usepackage{enumerate}			% Nummerierte Listen erstellen

\usepackage{colortbl}

%%%%%%% Mathematik %%%%%%%%%

\usepackage{amsmath} 			% Mathematik-Paket

\usepackage{tikz}				% Pfeile und Linien

\usepackage{siunitx}			% Si-einheiten per kurzeingabe

\sisetup{
	locale = DE ,				% Si-Einheiten auf Deutsch
	per-mode = symbol
}


%%%%%% Verlinkungen %%%%%%%%

\usepackage[pdftex,pdfauthor={\name},pdftitle={Arbeit von \name}]{hyperref} %Verlinkungen im Dokument, Metadaten anpassen

\usepackage{url}				% Verlinkungen auf Webseiten


%%%%%% Literatur %%%%%%%%

\usepackage{lipsum}				% Lorem Ipsum Dummy-Text

%\usepackage[backend=bibtex,style=authoryear, dashed=false]{biblatex}
\usepackage[backend=bibtex,style=numeric]{biblatex}  %% Literaturverwaltung
\bibliography{Literatur/Literatur.bib}	%% Literaturdatei aus welcher die Quellen geladen werden


%%%%%% Chemie %%%%%%%

%\usepackage[version=4]{mhchem}							% Paket für chemische Gleichungen
%\usepackage{expl3}										% Benötigt für mhchem
%\usepackage{calc}										% Benötigt für mhchem
%\RequirePackage{tikz}\usetikzlibrary{arrows.meta}		% Benötigt für mhchem (andere Pfeile)
%\usepackage{chemfig}									% Paket für chemische Figuren

%%%%%% Zusätzlich %%%%%%%
\usepackage{placeins} %Ausrichtung von Bildern

\usepackage{hyperref}  % PDA/A
\usepackage[a-1b]{pdfx}

%%%%%%%%%%%%%%%%%%%%%%% Die oben erstellten Befehle werden im Folgenden in das Titelblatt eingefügt.
%%%%% Titelblatt %%%%%% Bei abweichenden Vorstellungen können die entsprechenden Zeilen gelöscht
%%%%%%%%%%%%%%%%%%%%%%% und händisch selbst eingetragen werden


\title{Verbesserung einer existierenden Tracking-Software zur Steuerung eines Menschenmodells
	 in einer interaktiven Virtual Reality Umgebung
}  %% Titelüberschrift 

\subtitle{Improving an existing tracking software for controlling a human model in an
	 interactive virtual reality environment
} %% Untertitel

\subsubtitle{Bachelor-Thesis
			\newline
			Name: \hspace{21.3mm} \name 
			\newline 
			 Studiengang: \hspace{9.5mm} B.Sc. Informatik
			 \newline 
			 Matrikelnummer: \hspace{2.2mm} 2561916} %% Unteruntertitel

%\settitlepicture{Bilder/Titelbild} %% Titelbild laden

\setinstitutionlogo{Bilder/A999_DIKLogo}

%%%%%%%%%%%%%%%%%%%%%%% Hier beginnt das eigentliche Dokument
%%%%% Dokument   %%%%%%
%%%%%%%%%%%%%%%%%%%%%%%
% TODO: Pagenumbering
\begin{document}											% Dokumenten beginn
%--------------------------------------------------------------------------------------------------	
\maketitle		
\pagenumbering{gobble}										% Titelblat
\vspace*{\fill}
\begin{flushleft}
	\textbf{Elvir Sinancevic}
	\newline
	\textbf{Matrikelnummer: 2561916}
	\newline
	\textbf{Studiengang: Informatik B.Sc.}
	\newline\newline
	\textbf{Bachelor-Thesis}
	\newline
	\textbf{Thema: „Verbesserung einer existierenden Tracking-Software zur Steuerung eines Menschmodells in einer interaktiven Virtual Reality Umgebung"}
	\newline\newline
	Eingereicht: TODO DATUM
	\newline\newline
	Betreuer: Yübo Wang, M.Sc. M.A.
	\newline\newline
	Prof. Dr.-Ing. Reiner Anderl
	\newline
	Fachgebiet Datenverarbeitung in der Konstruktion
	\newline
	Fachbereich Maschinenbau
	\newline
	Technische Universität Darmstadt
	\newline
	Otto-Berndt-Straße 2
	\newline
	D-64287 Darmstadt	
\end{flushleft}
						% Seite mit Name, Matrikelnr, etc.
%--------------------------------------------------------------------------------------------------
\chapter*{Erklärung}\label{cha:Erklärung}
\textbf{Erklärung zur Abschlussarbeit gemäß § 22 Abs. 7 APB TU Darmstadt}
\newline\newline
Hiermit versichere ich, Elvir Sinancevic, die vorliegende Master-Thesis / Bachelor-Thesis gemäß § 22 Abs. 7 APB der TU Darmstadt ohne Hilfe Dritter und nur mit den angegebenen Quellen und Hilfsmitteln angefertigt zu haben. Alle Stellen, die Quellen entnommen wurden, sind als solche kenntlich gemacht worden. Diese Arbeit hat in gleicher oder ähnlicher Form noch keiner Prüfungsbehörde vorgelegen. 
\newline\newline
Mir ist bekannt, dass im Falle eines Plagiats (§38 Abs.2 APB) ein Täuschungsversuch vorliegt, der dazu führt, dass die Arbeit mit 5,0 bewertet und damit ein Prüfungsversuch verbraucht wird. Abschlussarbeiten dürfen nur einmal wiederholt werden.
\newline\newline
Bei einer Thesis des Fachbereichs Architektur entspricht die eingereichte elektronische Fassung dem vorgestellten Modell und den vorgelegten Plänen.
\newline\newline
\newline\newline
\textbf{Thesis Statement pursuant to § 22 paragraph 7 of APB TU Darmstadt}
\newline\newline
I herewith formally declare that I, Elvir Sinancevic, have written the submitted thesis independently pursuant to § 22 paragraph 7 of APB TU Darmstadt. I did not use any outside support except for the quoted literature and other sources mentioned in the paper. I clearly marked and separately listed all of the literature and all of the other sources which I employed when producing this academic work, either literally or in content. This thesis has not been handed in or published before in the same or similar form.
\newline\newline
I am aware, that in case of an attempt at deception based on plagiarism (§38 Abs. 2 APB), the thesis would be graded with 5,0 and counted as one failed examination attempt. The thesis may only be repeated once.
\newline\newline
For a thesis of the Department of Architecture, the submitted electronic version corresponds to the presented model and the submitted architectural plans.
\newline\newline
\newline\newline
\parbox{5cm}{\centering Darmstadt, 27.04.2020\hrule
	\strut \centering\footnotesize Ort, Datum} \hfill\parbox{5cm}{\hrule
	\strut \centering\footnotesize Elvir Sinancevic}							% Erklärung
%--------------------------------------------------------------------------------------------------
\chapter*{Kurzfassung}\label{cha:Kurzfassung}

\lipsum[6]
\newline\newline
\textbf{Schlüsselwörter:} Industrie 4.0, virtuelle Realität, Menschmodell, Interaktion

%--------------------------------------------------------------------------------------------------
\begingroup
\let\clearpage\relax
\chapter*{Abstract}\label{sec:Abstract}

\lipsum[6]
\newline\newline
\textbf{Keywords:} Industry 4.0, Virtual Reality, human model, interaction

\endgroup							% Erklärung
%--------------------------------------------------------------------------------------------------	
\cleardoublepage
\pagenumbering{roman}
\tableofcontents											% Inhaltsverzeichnig
\cleardoublepage
%--------------------------------------------------------------------------------------------------
\phantomsection
\addcontentsline{toc}{chapter}{Abbildungsverzeichnis}
\listoffigures												% Abbildungsverzeichnis
\cleardoublepage
%--------------------------------------------------------------------------------------------------	
%\listoftables
%\addcontentsline{toc}{chapter}{Tabellenverzeichnis}			% Tabellenverzeichnis
%--------------------------------------------------------------------------------------------------	
\phantomsection
\chapter*{Abkürzungen} \label{cha:Abkürzungen}				% Abkürzungen
\addcontentsline{toc}{chapter}{Abkürzungen}					
\cleardoublepage
%--------------------------------------------------------------------------------------------------
\pagenumbering{arabic}
%--------------------------------------------------------------------------------------------------
\chapter{Einleitung} \label{cha:Einleitung}	

\lipsum[1]

%--------------------------------------------------------------------------------------------------
\section{Motivation}\label{sec:Motivation}

\lipsum[1]

%--------------------------------------------------------------------------------------------------
\section{Zielsetzung}\label{sec:Zielsetzung}

\lipsum[1]
bla bla bla bla bla bla bla bla bla bla bla bla bla bla bla bla bla bla bla bla bla
bla bla bla bla bla bla bla bla bla bla bla bla bla bla bla bla bla bla bla bla bla
bla bla bla bla bla bla bla bla bla bla bla bla bla bla bla bla bla bla bla bla bla
bla bla bla bla bla bla bla bla bla bla bla bla bla bla bla bla bla bla bla bla bla

%--------------------------------------------------------------------------------------------------
\section{Aufbau der Arbeit}\label{sec:AufbauDerArbeit}
Der Aufbau dieser Arbeit ist in Abbildung \ref{fig:AufbauDerArbeit} dargestellt.
\begin{figure}[h]
	\centering
	\includegraphics[width=1\linewidth]{Bilder/A55_AufbauNeu}
	\caption{Der Aufbau der Arbeit, eigene Abbildung}
	\label{fig:AufbauDerArbeit}
\end{figure}
\newline
Zunächst wird, nach der Einleitung, in Kapitel XX der aktuelle Stand der Technik thematisiert. Konkret gibt es zu Beginn des Kapitels einen Einblick in die Geschichte der industriellen Revolutionen und der Wandlungen im Bereich der Informations- und Kommunikationstechnik (Kapitel XX), bevor anschließend die Potenziale (Kapitel XX) und Herausforderungen (Kapitel XX) von Industrie 4.0 erläutert werden. Schließlich gibt es noch eine Einführung in zwei Leitfäden, die Unternehmen bei dem Wandel zu Industrie 4.0 unterstützten sollen (Kapitel XX).
\newline
Im darauf folgenden Kapitel XX wird das übergeordnete Konzept erläutert. Aufgrund dessen wird zunächst der Prozess von der physischen Produktionsanlage bis hin zu ihrem virtuellen Klon erläutert (Kapitel XX) und die Arbeit in diesem Prozess eingeordnet. Daraufhin gibt es einen Überblick über die Mensch-Maschine-Interaktion im Kontext dieser Arbeit (Kapitel XX), bevor auch in diesem Kapitel die Arbeit entsprechend eingeordnet wird. Anschließend wird das eigentliche Konzepts des Menschmodells aufgebaut (Kapitel XX), indem zunächst die Anforderungen an das Menschmodell und die Interaktionsschnittstelle erläutert werden (Kapitel XX). Des Weiteren gibt es eine Einführung in das Functional-Mockup Interface (Kapitel XX), bevor erläutert wird, wie die Umsetzung des Menschmodells mit Hilfe des FMI Standards auszusehen hätte (Kapitel XX) und worin die Potenziale liegen würden (Kapitel XX).
\newline
Aufbauend auf den in Kapitel XX gewonnen Verständnis und insbesondere unter Berücksichtigungen der in Kapitel XX gestellten Anforderung wird in Kapitel XX die eigentliche Umsetzung des Konzepts mit Hilfe der Unity Engine erläutert. Aufgrund dessen gibt es zunächst eine Einführung in die Verwendete Hardware (Kapitel XX) und Software (Kapitel XX), bevor Anschließend die eigentliche Umsetzung des Menschmodells (Kapitel XX) und der Interaktionsschnittstelle (Kapitel XX) erläutert werden.
\newline
In Kapitel XX gibt es zunächst eine Einführung in die Vorgehensweise beim Einbinden des Menschmodells und der Interaktionsschnittstelle in ein neues Projekt (Kapitel XX), bevor anschließend die Vorgehensweise beim Einfügen neuer Objekte in der Umgebung erläutert wird (Kapitel XX). Schließlich werden in Kapitel XX und XX die in Kapitel XX gestellten Anforderungen an das Menschmodell und die Interaktionsschnittstelle, unter Berücksichtigung der in Kapitel XX erläuterten Umsetzung des Konzepts, validiert.
\newline
Abschließend gibt es in Kapitel XX einen Ausblick (Kapitel XX), der sich damit befasst wie die Zukunft von Industrie 4.0, insbesondere im Hinblick auf Virtual Reality, aussehen könnte. Des Weiteren werden die wichtigsten Erkenntnisse der Arbeit zusammengefasst (Kapitel XX).

%--------------------------------------------------------------------------------------------------						% Einleitung
%--------------------------------------------------------------------------------------------------	
%--------------------------------------------------------------------------------------------------
\chapter{Stand der Technik}\label{cha:StandDerTechnik}

\lipsum[2]

%--------------------------------------------------------------------------------------------------
\section{Industrie 4.0}\label{sec:Industrie4.0}
Gliederung für den Abschnitt Industrie 4.0 ist fast gleich zur Gliederung von Jonathan Klein
\newline

\lipsum[2]

\subsection{Die Geschichte industrielle Revolution}\label{sec:IndustrielleRevolution}

\lipsum[2]

\subsection{Potentiale von Industrie 4.0}\label{sec:PotentialeIndustrie4.0}

\lipsum[2]

\subsection{Herausforderungen bei der Umsetzung}\label{sec:HerausforderungenUmsetzung}

\lipsum[2]

\subsection{Leitfaden für Industrie 4.0}\label{sec:LeitfadenUmsetung}

\lipsum[2]

%--------------------------------------------------------------------------------------------------
\section{Mensch-Computer Interaktion}\label{sec:HCI}

\lipsum[2]

\subsection{Die Geschichte der Mensch-Computer Interaktion}\label{sec:HCIGeschichte}

\lipsum[2]

\subsection{Relevanz für meine Arbeit}\label{sec:RelevanzHCI}

\lipsum[2]

\subsection{???}\label{sec:???}

\lipsum[2]

%--------------------------------------------------------------------------------------------------
\section{Virtual Reality}\label{sec:VR}

\lipsum[2]

\subsection{Die Geschichte von Virtual Reality Hardware}\label{sec:VRGeschichte}

\lipsum[2]

\subsection{Stand der Technik}\label{sec:VRStandDerTechnik}

\lipsum[2]

\subsection{Technische Herausforderungen}\label{sec:VRHerausforderungen}
\lipsum[2]

\subsection{Potential und Ausblick}\label{sec:VRPotentialUndAusblick}

\lipsum[2]

\subsection{????}\label{sec:????}

\lipsum[2]

%--------------------------------------------------------------------------------------------------					% Stand der Technik
%--------------------------------------------------------------------------------------------------	
%--------------------------------------------------------------------------------------------------
\chapter{Aufbau des Konzepts}\label{cha:AufbauDesKonzepts}

In diesem Kapitel wird der Aufbau des Konzepts erklärt. Daher gibt es zuerst einen Einblick in den Digitalisierungsprozess von Produktionseinlagen. Daraufhin wird diese Arbeit in das Gesamtkonzept der Mensch-Maschine Interaktion eingeordnet. Anschließend wird ein Modell mit Hilfe des Functional-Mockup-Interfaces aufgestellt. Schließlich gibt es einen Einblick in die Potenziale.

%--------------------------------------------------------------------------------------------------
\section{Von der physischen Produktionsanlage zum virtuellen Klon}\label{sec:PhysischZumKlon}

Um die Produktionsabläufe einer Fabrik in einer virtuellen Welt abzubilden gibt es insgesamt sieben Schritte die befolgt werden müssen.

\begin{enumerate}
	\item \textbf{Scannen der Fabrik} \\
	Im ersten Schritt werden die Produktionsräume inklusive der Produktionsanlagen gescannt. Das Ergebnis dieses Scans ist eine Punktwolke.
	\item \textbf{Umwandlung des Scans} \\
	Der Scan muss von einer Punktewolke in CAD Modelle umgewandelt werden. Da die CAD Modelle zu groß und daher unpraktikabel für die Entwicklungsumgebung Unity sind, werden diese in das OBJ Format umgewandelt. Die Modelle sind für den späteren Aufbau des virtuellen Klons der echten Produktionsanlage notwendig.
	\item \textbf{Abbildung: Produktionsanlage $\rightarrow$ Modell} \\
	Es wird ein physisches Modell zur Darstellung einiger wichtiger Herausforderungen der echten Produktionsanlage gebaut. Dies könnte Beispielsweise die Form einer Produktionsstraße mit mehreren Stationen besitzen.
	\item \textbf{Virtueller Aufbau} \\
	Das physische Modell der echten Produktionsanlage aus Schritt drei wird in der virtuellen Welt abgebildet (virtueller Klon) und mit dem Menschmodell begehbar gemacht. Dieser Schritt erlaubt die virtuelle Planung einer Produktionsanlage.
	\item \textbf{Konnektivität und Kommunikation} \\
	Die Produktionsanlagen des physischen Modells werden mit der Cloud vernetzt um eine bidirektionale Kommunikation zwischen dem physischem Modell und dem virtuellen Klon zu ermöglichen.
	\item \textbf{Interaktion} \\
	Die Interaktion mit zwischen dem Bediener und den Produktionsanlagen in der virtuellen Welt wird implementiert. Es müssen unter Umständen anwendungsspezifische Benutzeroberflächen implementiert werden.
	\item \textbf{Skalierung} \\
	Skalierung der Vorgehensweise durch Anwendung von Schritt vier bis sechs auf die echte Produktionsanlage.
\end{enumerate}
Durch das Einhalten dieser Schritte erhält man einen virtuellen Klon der echten Produktionsanlage. Diesen virtuellen Klon kann man für viele Zwecke einsetzen, dazu gehören Beispielsweise die Produktionsplanung oder das Fernsteuern von Produktionsanlagen. Das Ergebnis dieser Arbeit kommt bei Schritt vier und sechs zum Einsatz [Quelle Yübo?].

%--------------------------------------------------------------------------------------------------
\section{Mensch-Maschine Interaktion im Kontext dieser Arbeit}\label{sec:MMInteraktion}
\begin{figure}[h]
	\centering
	\includegraphics[width=0.7\linewidth]{Bilder/A19_MMI}
	\caption{Mensch-Maschine Interaktion im Kontext dieser Arbeit, eigene Abbildung}
	\label{fig:MMI}
\end{figure}
\noindent Durch die Abbildung der echten Produktionsanlage in der virtuellen Welt findet die Mensch-Maschine-Interaktion in zwei Schritten statt. Im ersten Schritt interagiert der Mensch mit der Software (dem Computer) über VR-Hardware. Dieser Computer interagiert dann über eine Cloud mit der echten Produktionsanlage. Die Kommunikation findet in beiden Schritten bidirektional statt.
\newline\newline
Diese Arbeit befasst sich mit dem ersten Schritt der oben erklärten Mensch-Maschine Interaktion. Für dem Rest dieser Arbeit werden wir annehmen, dass es bereits eine geeignete Infrastruktur für die Kommunikation zwischen dem Computer und der Produktionsanlage über eine Cloud gibt.

%--------------------------------------------------------------------------------------------------
\section{Das Modell der Interaktivität mit einem Menschmodell}\label{sec:ModellAufbau}
Basierend auf den im folgenden definierten Anforderungen soll mit Hilfe des Functional-Mockup Interfaces ein Modell aufgebaut werden.

\subsection{Anforderungen an das Modell}\label{sec:AnforderungenKonzept}
Da das Modell aus zwei Teilen besteht, haben beide Teile Ihre eigenen Anforderungen. Diese werden im Folgenden erläutert. Das Ziel ist es ein Menschmodell zu schaffen, welches interaktiv mit seiner Umgebung interagieren kann.

\subsubsection{Menschmodell}\label{sec:AnforderungenMensch}
\begin{enumerate}
	\item \textbf{Genauigkeit} \\
	Die Genauigkeit der Abbildung der Tracking-Daten auf das virtuelle menschliche Abbild ist von zentraler Bedeutung. Vor allem in einem Szenario, bei dem sich mehrere virtuelle menschliche Abbildungen in einem virtuellen Klon einer Produktionsanlage aufhalten ist die möglichst genaue Abbildung unabdingbar, um Beispielsweise auf andere Bediener in der virtuellen Welt Rücksicht nehmen zu können.
	\item \textbf{Echtzeit} \\
	Sowohl das Tracking des Bedieners als auch die Abbildung der Tracking-Daten auf das virtuelle menschliche Abbild sind echtzeitkritische Anwendungen. Die Verzögerungen müssen möglichst gering sein um eine gute Nutzbarkeit zu ermöglichen.
	\item \textbf{Interoperabilität} \\
	Das Menschmodell muss umgebungsunabhängig implementiert werden. Dies bedeutet, dass das Menschmodell in beliebigen Umgebungen (Fabriken, Produktionshallen, etc.) eingesetzt werden kann.
	\item \textbf{Modularität} \\
	Sowohl einzelne Komponenten als auch das ganze Menschmodell an sich unterliegen der Anforderung der Modularität, um in Zukunft Verbesserungen oder Erweiterungen am Modell durchführen zu können.
\end{enumerate}

\subsubsection{Interaktion}\label{sec:AnforderungenInteraktion}
\begin{enumerate}
	\item \textbf{Bidirektionalität} \\
	Der Informationsaustausch muss bidirektional stattfinden. Das bedeutet, dass der Bediener über geeignete, leicht zu verstehende, optisch ansprechende und vor allem einheitliche Benutzeroberflächen interagieren kann. Einerseits ermöglichen diese Benutzeroberflächen dem Bediener das Interagieren mit Produktionsanlagen, andererseits werden auf diesen Benutzeroberflächen für den Bediener relevante Informationen in geeigneter Form dargestellt.
	\item \textbf{Genauigkeit} \\
	Auch bei der Interaktion spielt die Genauigkeit eine wichtige Rolle. Der Bediener sollte intuitiv, mit relativ geringen Lernaufwand, schnell und vor allem präzise mit der Umgebung interagieren können.
	\item \textbf{Echtzeit} \\
	Die Interaktion muss mit einer möglichst geringen Verzögerung stattfinden, um dem Nutzer schnelle Reaktionen und spontane Anpassungen oder Verbesserungen zu ermöglichen.
	\item \textbf{Interoperabilität} \\
	Die Interaktion muss über eine einheitliche Schnittstelle stattfinden um Interoperabilität zu gewährleisten, sodass sich die Funktionalität auf beliebige Maschinen und Produktionsanlagen übertragen lässt.
	\item \textbf{Modularität} \\
	Durch das Gewährleisten der Modularität wird sichergestellt, dass die Interaktion in Zukunft auch auf neue VR-Hardware übertragen werden kann ohne das die gesamte Software neu geschrieben werden muss.
\end{enumerate}
Durch das Einhalten dieser Herausforderungen ergibt sich ein Menschmodell, welches in Echtzeit die Bewegungen des Bedieners auf das virtuelle Menschliche Abbild projiziert. Des Weiteren ermöglicht das Modell dem Bediener in der virtuellen Welt mit dem virtuellen Klon der Produktionsanlage zu interagieren. Da von Anfang an ein Fokus auf Interoperabilität und Modularität gelegt wird, ist das ganze Modell erweiterbar und an verschiedene Hardware und unterschiedliche Einsatzzwecke anpassbar.

\subsection{Das Functional-Mockup Interface}\label{sec:DasFMU}
Das Functional-Mockup Interface (FMI) ist ein ursprünglich von der Daimler AG entwickelter unabhängiger Standard für den Austausch von dynamischen Modellen und Co-Simulation. Zum Einsatz kommt das Functional-Mockup Interface beispielsweise beim Modellaustausch oder der Co-Simulation zwischen Zulieferern und den OEMs (Original Equipment Manufacturer) [24, FMI-Paper, S.1].
Anfang 2010 wurde die erste Version (Version 1.0) des FMI Standards veröffentlicht, welche zunächst nur den Modellaustausch unterstützte. Einige Monate später folgte dann die Unterstützung für Co-Simulation. Mittlerweile gab weitere Nachfolger Versionen. Die aktuellste Version (Version 2.0.1) wurde im Oktober 2019 veröffentlicht, unterscheidet sich aber von der Vorgängerversion (Version 2.0) aus dem Jahr 2014 nur durch einige Bugfixes [2, FMI-Spez, S.2].
Da es neben den anwendungsspezifischen Standards (z.B. Matlab/Simulink: S-Functions) zunächst keinen unabhängigen Standard für den Austausch von Modellen oder die Co-Simulation gab [24, FMI-Paper, S.1], liegt der größte Vorteil des FMI-Standards in seiner Unabhängigkeit.
\newline
Wie bereits erwähnt gibt es für den FMI Standard zwei übergeordnete Anwendungsszenarien:
\begin{enumerate}
	\item \textbf{Modellaustausch} \\
	Der FMI Standard ermöglicht Modellierungsumgebungen ein dynamisches System Modell in C-Code zu repräsentieren, sodass dieses Modell auch in anderen Modellierungsumgebungen genutzt werden kann. Ermöglicht wird dies durch die Darstellung der Modelle mit Hilfe von mathematischen Gleichungen (z.B. Algebraische Gleichungen, Differentialgleichungen oder Diskrete Gleichungen), welche auch Zeit-, Zustands- oder Schritt-Events enthalten können. Es werden große Modelle und sowohl die offline als auch die online Simulation unterstützt [25, FMI-Spez, S.4].
	\item \textbf{Co-Simulation} \\
	Des Weiteren ermöglicht der FMI Standard das Verbinden von mehreren Subsystem, welche über vordefinierte Kommunikationsschnittstellen kommunizieren und somit eine Co-Simulationsumgebung schaffen. Dabei wird der Datenaustausch und die Synchronisation zwischen den einzelnen Subsystemen durch einen sogenannten Master Algorithmus gesteuert. Bei den Master Algorithmen ist zu beachten, dass diese kein Teil des eigentlichen FMI Standards sind [25, FMI-Spez, S.4]. Daher sind die Master Algorithmen nicht standardisiert und können unterschiedliche zusätzliche Funktionalitäten wie z.B. den Austausch von Variablen unterstützen [24, FMI-Paper, S.1].
\end{enumerate}
Die grundlegenden Komponenten die beim Einsatz des FMI Standards immer zum Einsatz kommen sind das „FMI Application Programming Interface (C)“ und das „FMI Description Schema (XML)“. Ersteres enthält dabei alle benötigten Gleichungen des Models. Dabei kommt die Programmiersprache C zum Einsatz, da sie portabel ist und auch für den Einsatz in eingebetteten System geeignet ist. Die XML-Datei (vgl. Abbildung \ref{fig:FMIOverview}) hingegen enthält die Definitionen aller Variablen in einem standardisierten Format. Diese Variablendefinition ist durchaus eine komplexe Datenstruktur. Die Simulationstools können selber entscheiden wie sie mit diesen Daten umgehen und wie sie die Daten in ihren Programmen repräsentieren möchten. Dieser Ansatz erlaubt das Speichern und Laden der Variablendefinitionen ohne Einbußen bei der Performance in der Programmiersprache der Simulationsumgebung, wie z.B. C\# oder Java [25, FMI-Spez, S8].
\begin{figure}[h]
	\centering
	\includegraphics[width=0.7\linewidth]{Bilder/A20_FMIOverview}
	\caption{Aufbau der FMI Variablenbeschreibung [25, FMI-Spez, S.30]}
	\label{fig:FMIOverview}
\end{figure}
\newline
Da das FMI nur ein Interface ist wird eine Instanz des Interfaces FMU (Functional-Mockup Unit) genannt. Das FMI Interface soll den Einsatz von FMUs in beliebigen Simulationsumgebungen einfach gestalten. Dafür werden durch die Simulationsumgebung Funktionen des FMI aufgerufen um ein FMU zu generieren. Es ist zusätzlich anzumerken, dass eine FMU beliebig oft und sogar als Teil von anderen Modellen generiert werden kann [25, FMI-Spez., S.8]. Eine FMU wird in einer ZIP-Datei mit der Endung „.fmu“ bereitgestellt und kann neben den bereits angesprochenen Komponenten (C-Code und XML-Datei) zusätzliche Anwendungsspezifische Daten wie Icons oder Dokumentationen enthalten [25, FMI-Spez, S.4+9] In Abbildung \ref{fig:FMIBlock} ist eine Instanz einer FMU illustriert. Eine FMU enthält neben den Eingabeparametern (Rot) und den Ausgabeparametern (Blau) noch interne Werte. Zu den internen Werten gehört die Zeit t, interne Parameter p und nach außen sichtbare Variablen v [25, FMI-Spez, S.8].
\begin{figure}[h]
	\centering
	\includegraphics[width=0.5\linewidth]{Bilder/A21_FMIBlock}
	\caption{FMU Instanz [25, FMI-Spez, S.9]}
	\label{fig:FMIBlock}
\end{figure}
\newline
In der Abbildung \ref{fig:FMUEinordnung} ist noch einmal dargestellt, wie das Zusammenspiel zwischen der Simulationsumgebung, der FMU und dem FMI aussieht. Die Simulationsumgebung liest Variablendefinition (XML) der FMU, welche dem FMI Standard entspricht. Dies ermöglicht der Simulationsumgebung eine oder mehrere Instanzen der FMU zu generieren. Aufgrund dieser Struktur kann jede Simulationsumgebung eine eigene Benutzeroberfläche einsetzen und selber entscheiden wie die Daten repräsentiert werden.
\begin{figure}[h]
	\centering
	\includegraphics[width=1\linewidth]{Bilder/A22_User-FMU-FMI}
	\caption{Funktionsweise der FMU in der Simulationsumgebung [26, QTronic, S.7]}
	\label{fig:FMUEinordnung}
\end{figure}

\subsection{Der Aufbau des Models}\label{sec:ModelAufbau}
Um die in Kapitel \ref*{sec:AnforderungenKonzept} erläuterten Herausforderungen umzusetzen soll mit Hilfe des in Kapitel \ref*{sec:DasFMU} eingeführten FMI Standards ein theoretisches Modell geschaffen werden, wie ein Menschenmodell mit der Möglichkeit mit der Umgebung zu interagieren aufgebaut sein könnte. Daher wird im Folgenden vorgestellt, wie die Variablendefinition im XML-Format und das eigentliche Modell in C-Code auszusehen hätten.

\subsubsection{Variablendefinition (XML)}\label{sec:Variablendefinition}
Wie in Abbildung \ref{fig:FMIOverview} dargestellt besteht die XML-Datei aus mehreren Komponenten. Nicht alle Komponenten sind für jeden Einsatzzweck relevant, daher werden im Folgenden die relevanten Komponenten genauer erläutert.

\paragraph{Attributes}\label{sec:AttributeFMU}
\noindent Neben den in Abbildung \ref{fig:FMIOverview} dargestellten Elementen, enthält die XML-Datei noch das Element „Attributes“. Es gibt insgesamt 12 mögliche Attribute, jedoch sind nur vier zwingend notwendig [25, FMI-Spez, S.33].
\begin{enumerate}
	\item \textbf{fmiVersion} \\
	Dieses Attribut gibt an, welche Version des FMI Standards bei der Generierung der XML-Datei
	eingesetz wurde [25, FMI-Spez, S.33].
	\item \textbf{modelName} \\
	Durch dieses Attribut wird dem Modell ein Name zugewiesen [25, FMI-Spez, S.33].
	\item \textbf{guid} \\
	Durch das Attribut „guid“ (Globally Unique Identifier) wird sichergestellt, dass die XML-Datei 
	Kompatibel ist mit den C-Funktionen der FMU. Dafür wird in der Regel eine Art Fingerabdruck
	generiert und in den C-Funktionen hinterlegt [25, FMI-Spez, S.33].
	\item \textbf{variableNamingConvention} \\
	Dieses Attribut bestimmt ob die Namen der variablen einer bestimmten Konvention 
	unterliegen. Standardmäßig wird hier der Token „flat“ verwendet, wenn alle Variablen
	in einer Liste von Strings vorliegen [25, FMI-Spez, S.33].
	\item \textbf{description (optional)} \\
	In diesem Attribut kann auf Wunsch eine kurze Beschreibung der FMU hinterlegt werden [25, 
	FMI-Spez, S.33].
	\item \textbf{author (optional)} \\
	Dieses Attribut ermöglicht es Angaben über die Autoren (Entwickler) zu machen [25, FMI-
	Spez, S.33].
	\item \textbf{version (optional)} \\
	Um einen besseren Überblick über die verschiedenen Versionen einer FMU zu ermöglichen,
	kann in diesem Attribut die Versionsnummer (z.B. Version 1.0) hinterlegt werden [25, FMI-
	Spez, S.33].
	\item \textbf{copyright (optinal)} \\
	In diesem Attribut können Informationen über das Copyright hinterlegt werden [25, FMI-
	Spez, S.33].
	\item \textbf{license (optional)} \\
	Es besteht die Möglichkeit durch dieses Attribut angaben über die Lizenz für die Verwendung
	der FMU zu machen [25, FMI-Spez, S.33].
	\item \textbf{generationTool (optional)} \\
	In diesem Attribut kann der Name des Tools mit dem die XML-Datei generiert wurde
	hinterlegt werden [25, FMI-Spez, S.33].
	\item \textbf{generationDateAndTime (optional)} \\
	Dieses Attribut ermöglicht es das Datum und die genaue Uhrzeit der Generierung der XML-
	Datei zu hinterlegen. Dabei wird das Format „YYYY-MM-DDThh:mm:ssZ“ verwendet. Das
	„T“ soll lediglich das Datum von der Uhrzeit trennen und das „Z“ steht für die Zeitzone
	(Greenwich Meantime) [25, FMI-Spez, S.33].
	\item \textbf{numberOfEventIndicators (optional)} \\
	Falls die FMU für Co-Simulation verwendet wird, wird dieses Attribut ausgelassen. Ansonsten
	wird hier die Anzahl der Indikatoren für Events beim Einsatz der FMU für Modellaustausch
	hinterlegt [25, FMI-Spez, S.33].
\end{enumerate}
Abbildung \ref{fig:FMUAttribute} illustriert, wie die Attribute für die FMU aussehen könnten.
\begin{figure}[h]
	\centering
	\includegraphics[width=1\linewidth]{Bilder/A24_FMUAttributBeispiel}
	\caption{Darstellung der Attribute für das Menschmodell, eigene Abbildung}
	\label{fig:FMUAttribute}
\end{figure}

\paragraph{ModelExchange / Co-Simulation}\label{sec:ModellExchangeCoSimulation}
\noindent Dieses Element gibt an, ob die FMU für Modellaustausch und/oder Co-Simulation ausgelegt ist.
Falls das Element „ModelExchange“ vorhanden ist, enthält die FMU entweder selbst das eigentliche Modell oder die Kommunikationsschnittstelle zu einem Tool, welches das Modell enthält. Die Simulationsumgebung kümmert sich dann um die Simulation.
Falls jedoch das Element „CoSimulation“ vorhanden ist, enthält die FMU selbst das Modell und kümmert sich selbst um die Simulation, oder enthält die Kommunikationsschnittstelle zu einem Tool welches das Modell enthält und sich selbst um die Simulation kümmert. Die Simulationsumgebung enthält den in Kapitel \ref*{sec:DasFMU} bereit angesprochenen Master Algorithmus, um mehrere FMUs gemeinsam in einer Co-Simulationsumgebung einzusetzen  [25, FMI-Spez, S.30].
Es ist noch wichtig anzumerken, dass mindestens eins der beiden Elemente vorhanden sein muss, um den Typ der FMU zu definieren [25, FMI-Spez, S.31].
Da das Menschmodell sowohl das Modell enthält als auch die Simulation (hier: Abbildung der Bewegungsdaten auf das virtuelle menschliche Abbild) eigenständig übernimmt, müsste die FMU auf Co-Simulation ausgelegt werden [25, FMI-Spez, S.30].
\newline
In Abbildung \ref{fig:FMUCoSimulation} ist illustriert, wie die FMU als Teil einer Anwendung funktionieren würde.
\begin{figure}[h]
	\centering
	\includegraphics[width=1\linewidth]{Bilder/A23_FMUCoSimulation}
	\caption{Funktionsweise der FMU als Teil einer Anwendung, eigene Abbildung}
	\label{fig:FMUCoSimulation}
\end{figure}
%
%\paragraph{UnitDefinitions}\label{sec:1}
%\noindent bla bla blabla bla blabla bla blabla bla blabla bla blabla bla blabla bla blabla bla 
%
%\paragraph{TypeDefinitions}\label{sec:2}
%\noindent bla bla blabla bla blabla bla blabla bla blabla bla blabla bla blabla bla blabla bla 
%
%\paragraph{LogCategories}\label{sec:3}
%\noindent bla bla blabla bla blabla bla blabla bla blabla bla blabla bla blabla bla blabla bla 
%
%\paragraph{DefaultExperiment}\label{sec:4}
%\noindent bla bla blabla bla blabla bla blabla bla blabla bla blabla bla blabla bla blabla bla 
%
%\paragraph{VendorAnnotations}\label{sec:5}
%\noindent bla bla blabla bla blabla bla blabla bla blabla bla blabla bla blabla bla blabla bla 
%
\paragraph{ModelVariables}\label{sec:6}
\noindent In dem Bereich “ModelVariables“ sind die öffentlichen Variablen der FMU definiert. Diese Variablen werden unter dem übergeordneten Typ „ScalarVariable“ gespeichert und erhalten alle einen eindeutigen Index. Des Weiteren sind „Scalar Variables“ primitive Variablen, wie z.B. ganze Zahlen (Integer) oder Wahrheitswerte (Boolean) (Vgl. Abbildung \ref{fig:FMIOverview}). Zusätzlich zu dem Variablentyp sind noch optionale Annotationen (Vgl. Abbildung \ref{fig:FMIOverview}) und Attribute Bestandteil dieser Variablen [25, FMI-Spez, S.45]. Zu den Attributen gehören der Name (name), die sogenannte Wert-Referenz (valueReference), eine optionale Beschreibung (description), die Kausalität (causality), also ob es sich z.B. eine Input- oder Output-Variable handelt, die Variabilität (variability), welche beschreibt ob die Variable z.B. nicht Veränderbar ist und der Startwert (initial) [25, FMI-Spez, S.46-49].

\paragraph{ModelStructure}\label{sec:7}

\subsubsection{Aufbau des Codes}\label{sec:CCode}
In der Abbildung \ref{fig:CCodeDarstellung} ist ein stark vereinfachter Überblick über den Aufbau des Codes dargestellt. Die VR Hardware ist über das Programm VIVE Wireless mit der VR-Schnittstelle für Computer SteamVR verbunden. In dem Projekt ist zusätzlich das SteamVR Plugin für die Entwicklungsumgebung Unity installiert, welches die Schnittstelle zwischen dem Projekt und der VR-Hardware vollendet. Zur Verfügung gestellt wird das Modell von dem Plugin „Final IK“ (IK = Inverse Kinematics), welches gleichzeitig die Kalibrierung, die Berechnung der inversen Kinematik und die Abbildung der Bewegungsdaten auf das Modell übernimmt. Die wichtigsten Klassen sind die Klassen „Calibration Controller“ und „TrackerAssignment“. Die Klasse „Calibration Controller“ ruft zunächst die Methode für die Zuweisung der Tracker in der Klasse „TrackerAssignment“ auf, bevor der Kalibrierungsprozess gestartet wird. Durch die Klasse „TrackerAssignment“ werden nur die aktiven Tracker den entsprechenden Körperteilen (Füße, Knie, Steißbein, Hände, Ellenbogen, Kopf) zugewiesen. Die getrackten Körperteile enthalten das Skript „SteamVR Tracked Object“, über welches sie die Bewegungsdaten des Bedieners erhalten. Es ist anzumerken, dass das Menschenmodell noch über weitere Funktionalitäten, wie z.B. die Interaktion mit der Umgebung verfügt. Einen genaueren Einblick in den Aufbau des Projektes gibt es in Kapitel \ref{cha:Umsetzung}.
\begin{figure}[h]
	\centering
	\includegraphics[width=1\linewidth]{Bilder/A25_CCodeDarstellung}
	\caption{Aufbau des Codes (grobe Darstellung), eigene Abbildung}
	\label{fig:CCodeDarstellung}
\end{figure}

\subsection{Potenziale}\label{sec:PotenzialeFMU}
bla bla blabla bla blabla bla blabla bla blabla bla blabla bla blabla bla blabla bla blabla bla blabla bla blabla bla blabla bla blabla bla blabla bla blabla bla blabla bla blabla bla blabla bla bla
%--------------------------------------------------------------------------------------------------						% Menschmodell
%--------------------------------------------------------------------------------------------------	
%--------------------------------------------------------------------------------------------------
\chapter{Umsetzung des Konzepts mit Hilfe der Unity Enginge}\label{cha:Umsetzung}
In diesem Kapitel wird die Umsetzung des Menschmodells und der Interaktion in der Entwicklungsumgebung der Unity Engine vorgestellt. Daher gibt es zunächst einen Einblick in die genutzte Hardware und Software, bevor das Menschmodell und die Interaktionsschnittstelle genauer erläutert werden.
%--------------------------------------------------------------------------------------------------
\section{Eingesetzte Hardware}\label{sec:Hardware}
Für die Umsetzung dieses Projektes wurde die Virtual Reality Brille \textbf{VIVE Pro} (Vgl. Abbildung \ref{fig:ViveproKit}) vom Hersteller HTC verwendet, da sie eine der Leistungsstärksten Brillen auf dem Markt ist. Zu den Stärken dieser Brille gehören das kontrastreiche OLED-Display, die sehr hohe Auflösung von 1440 x 1600 Pixeln pro Auge, eine Bildwiederholrate von 90Hz, ein Sichtfeld von 110 Grad und vor allem die Möglichkeit die Brille mit Hilfe des \textbf{Vive Wireless Sets} (Vgl. Abbildung \ref{fig:WirelessKit}) kabellos zu verwenden. Es ist anzumerken, dass für das kabellose Verwenden dieser VR Brille eine \textbf{Erweiterungskarte} in den PC eingebaut werden muss um einen speziellen \textbf{Empfänger} für die Signale der Brille anzuschließen. Des Weiteren muss ein \textbf{Sender} an der VR-Brille angebracht werden, welcher mit dem am PC angeschlossenen Empfänger kommuniziert und durch einen \textbf{mobilen Akku} mit Strom versorgt wird. Der mitgelieferte mobile Akku ermöglicht einen kabellosen Einsatz der VR Brille für bis zu sechs Stunden (Vgl. Abbildung \ref{fig:WirelessKit}) [28, ViveWeb].
\newline\newline
Um die Brille zu verwenden bedarf es mindesten zwei der sogenannten \textbf{SteamVR 2.0 Basisstationen} (Vgl. Abbildung \ref{fig:ViveproKit}), welche dem Bediener in Kombination mit des Vive Wireless Adapters eine enorme Bewegungsfreiheit ermöglichen. Beim Einsatz von zwei solcher Basisstationen ist eine Raumgröße von bis zu 5m x 5m, also 25m² möglich. Es ist sogar möglich bis zu vier solcher Basisstationen zu Verwenden und somit eine Raumgröße von bis zu 10m x 10m, also 100m² zu unterstützen [28, ViveWeb].
\newline\newline
Ein weiteres notwendiges Zubehör der Brille sind die beiden \textbf{Controller} (Vgl. Abbildung \ref{fig:ViveproKit}), die es dem Bediener ermöglichen mit der virtuellen Umgebung zu interagieren. Beide Controller werden wie die VR Brille durch die Basisstationen im Raum geortet und liefern Informationen über ihre eigene Position und Ausrichtung im Raum. Zusätzlich verfügen beide Controller über jeweils fünf Tasten, welche man mit eigenen Funktionalitäten versehen kann. Es ist anzumerken, dass die Tasten alle unterschiedlich sind und daher für unterschiedliche Zwecke verwendet werden können [29, ViveController].
\newline
Auf der Vorderseite der Controller befinden sich insgesamt drei Tasten [29, ViveController]:
\begin{itemize}
	\item Die erste dieser Tasten (ganz unten) ist die Taste um das Hauptmenü aufzurufen. Diese 
	Taste wird in der Regel nicht überschrieben und behält diese Funktionalität bei.
	\item Die zweite Taste (mittig) ist gleichzeitig ein Berührungsempfindliches Trackpad. Dem 
	Entwickler steht es frei, ob er diese Taste als einfache Taste oder als Trackpad verwenden 
	möchte. Es ist sogar Möglich beide Funktionen gleichzeitig in einer Anwendung zu 
	unterstützen. Dadurch eröffnen sich viele Anwendungsmöglichkeiten für diese Taste.
	\item Die dritte Taste (oben) ist wiederrum eine ganz einfache Taste und wird in den meisten 
	Anwendungen als eine einfache Menü-Taste verwendet.
\end{itemize}
Auf der Rückseite und der Außenseite des Controllers befinden sich zwei weitere Tasten [29, ViveController]:
\begin{itemize}
	\item Die Tasten links und rechts an der Außenseite des Controllers bilden eine Taste, welche 
	ausgelöst wird, wenn der Bediener den Controller fest mit der Hand drückt.
	\item Die Taste auf der Rückseite hat wie das Trackpad auf der Vorderseite zwei 
	Einsatzmöglichkeiten. Sie kann einerseits als einfache Taste verwendet werden, andererseits 
	als berührungsempfindlicher Auslöser, da der Entwickler über die Software-Schnittstelle 
	auslesen kann wie tief die Taste eingedrückt wurde, ähnlich wie bei einem Gaspedal in
	einem Auto.
\end{itemize}
\begin{figure}[h]
	\centering
	\includegraphics[width=0.7\linewidth]{Bilder/A26_Vivepro}
	\caption{VIVE Pro (mitte), Controller und Basisstationen (außen) [A26]}
	\label{fig:ViveproKit}
\end{figure}
\begin{figure}[h]
	\centering
	\includegraphics[width=0.5\linewidth]{Bilder/A27_WirelessKit}
	\includegraphics[width=0.4\linewidth]{Bilder/A28_Vive+Wireless}
	\caption{Links: VIVE Wireless Set, inkl. Erweiterungskarte, Sender, Empfänger, Akku, Kabeln und Befestigungen. Rechts: VR Brille mit angeschlossenem Sender. [A27+A28]}
	\label{fig:WirelessKit}
\end{figure}
Neben den außerordentlich guten technischen Spezifikationen der HTC VIVE Pro waren die \textbf{HTC VIVE Tracker} (Vgl. Abbildung \ref{fig:ViveTracker}) ein weiterer Grund warum diese VR Brille zur Umsetzung dieser Arbeit ausgewählt wurde. Die VIVE Tracker werden genauso wie die VR Brille und die dazugehörigen Controller von den Basisstationen im raum geortet und liefern ebenfalls Informationen über ihre Position und Ausrichtung im Raum. Durch den kleinen Formfaktor können die Tracker an beliebigen Objekten befestigt werden um die Bewegung dieser Objekte in der virtuellen Welt abzubilden [30, ViveTracker].
\newline
\begin{figure}[h]
	\centering
	\includegraphics[width=0.5\linewidth]{Bilder/A29_ViveTracker}
	\caption{HTC VIVE Tracker [A29]}
	\label{fig:ViveTracker}
\end{figure}
\newline
Bei dieser Arbeit kamen die Tracker für die Ortung einzelner Körperteile zum Einsatz, da die Hände und der Kopf werden bereits durch die Controller und die VR Brille abgedeckt wurden. Konkret kamen die Tracker für die Ortung der Füße, der Knie, des Beckens und der Ellenbogen zum Einsatz. Durch das Schraubgewinde auf der Unterseite lassen sich die Tracker einfach befestigen. Für die Befestigungen am Becken und an den Füßen wurde auf \textbf{fertige Halterungen} zurückgegriffen (Vgl. Abbildung \ref{fig:Mounts}). Um die Tracker an den Knien und an den Ellenbogen zu befestigen habe ich mir \textbf{eigene Halterungen} gebaut (Vgl. Abbildung \ref{fig:Mounts}). Für diese Halterungen wurden handelsübliche Knie- und Ellenbogenschoner verwendet, durch die ein Loch gebohrt wurde um eine Schraube mit Hilfe einer Mutter zu fixieren. Durch das bereits erwähnte Schraubgewinde auf der Unterseite der Tracker ließen diese sich einfach an diesen Schrauben befestigen.
\begin{figure}[h]
	\centering
	\includegraphics[width=0.7\linewidth]{Bilder/A32_Mounts}
	\caption{Gekaufte (links) und eigene (rechts) Befestigungen für die Tracker, eigene Abbildung}
	\label{fig:Mounts}
\end{figure}

%--------------------------------------------------------------------------------------------------
\section{Eingesetzte Software}\label{sec:Software}
Wie in Abbildung \ref{fig:CodeDarstellung} bereits angedeutet kamen für die Umsetzung dieses Projektes verschiedene Anwendungen und Plugins zum Einsatz. Im Folgenden werden diese genauer erläutert:

\subsection{VIVE Wireless}\label{sec:VIVEWireless}
Die VIVE Wireless Anwendung (Vgl. Abbildung \ref{fig:VIVEWirelessSteamVR}) ist für die direkte Verbindung mit der VR Hardware verantwortlich. Wie bereits in Abbildung \ref{fig:WirelessKit} dargestellt wird dafür im Computer eine Erweiterungskarte installiert, die es einem ermöglicht den entsprechenden Empfänger für die Signale am PC anzuschließen. Der in Abbildung \ref{fig:WirelessKit} dargestellte an der VR Brille montierte Sender bildet das Gegenstück zu diesem Empfänger. Diese Hardware-Erweiterung in Kombination mit der VIVE Wireless Anbindung ermöglicht die kabellose Verwendung der VR Brille.

\subsection{SteamVR}\label{sec:SteamVR}
SteamVR (Vgl. Abbildung \ref{fig:VIVEWirelessSteamVR}) stellt eine weitere wichtige Anwendung im Kontext dieser Arbeit dar. Die meisten Anwendungen für VR Brillen von unterschiedlichen Herstellern sind auf die Schnittstelle der SteamVR Software ausgelegt. Das Gegenstück zu dieser Schnittstelle bildet das SteamVR Plugin, welches im weiteren Verlauf dieses Kapitels vorgestellt wird. Falls die VR Brille ohne das VIVE Wireless Set verwendet wird, wird diese über ein Kabel direkt mit dem PC und somit direkt mit der SteamVR Anwendung verbunden. Da für diese Arbeit das VIVE Wireless Set eingesetzt wurde, wird die VR Brille indirekt über die VIVE Wireless Anwendung mit der SteamVR Anwendung verbunden. Des Weiteren bringt die SteamVR Anwendung eine große Menge an Funktionalitäten mit sich. Dazu gehören Beispielsweise die Möglichkeit den Spielraum zu vermessen oder die Tastenbelegungen der Controller zu verändern. Zusätzlich bietet SteamVR eine Art Hauptmenü an, welches über die in Kapitel \ref{sec:Hardware} angesprochene Taste aufgerufen werden kann.
\newline
Zusammenfassend lässt sich sagen, dass SteamVR eine zentrale Schnittstelle bietet mit der sich die VR Brille, die Controller, die Basisstationen und die Tracker verwalten lassen.
\begin{figure}[h]
	\centering
	\includegraphics[width=0.8\linewidth]{Bilder/A33_VIVESteam}
	\caption{Links: VIVE Wireless, Rechts: SteamVR, eigene Abbildung}
	\label{fig:VIVEWirelessSteamVR}
\end{figure}

\subsection{Unity Engine}\label{sec:UnitEngine}
Die Unity Engine ist eine 3D-Entwicklungsumgebung und wurde für die Umsetzung dieser Arbeit verwendet. Als Programmiersprache für diese Entwicklungsumgebung kommt die Sprache C\# zum Einsatz.
\newline
Unity bietet den Vorteil in einem Projekt mehrere Szenen (Umgebungen) aufzusetzen und schnell zwischen diesen wechseln zu können. Des Weiteren können die Szenen direkt in der Anwendung bearbeitet werden wenn z.B. 3D-Modelle bearbeitet, hinzugefügt oder entfernt werden sollen. Des Weiteren bringt die Unity Engine eine Vielzahl von bereits eingebauten Funktionalitäten mit sich, die Entwicklern einiges an Arbeit ersparen können. So gibt es Beispielweise vorgefertigte Elemente für grafische Benutzeroberflächen oder sogar ein eingebautes Physiksystem mit dem sich z.B. Kollisionen von Objekten leicht abfragen können. Dank des eingebauten Asset Stores („Marktplatz“) lassen sich in einem internen Marktplatz dank der großen Entwickler-Community aus einer großen Auswahl an Erweiterungen, Texturen, Modelle, etc. beliebig viele Komponenten herunterladen und im eigenen Projekt einfügen. Aufgrund dessen kann man sagen, dass Unity als Entwicklungsumgebung den Entwicklern ermöglicht Modularität in Ihren Projekten umzusetzen.
\newline
Auch bei dieser Arbeit kamen Erweiterungen aus dem Asset Store zum Einsatz. Neben vereinzelten Texturen aus dem Asset Store sind die wichtigsten Erweiterungen das SteamVR und das Final IK Plugin, welche im Folgenden genauer erläutert werden.

\subsubsection{SteamVR Plugin für die Unity Engine}\label{sec:SteamVRPlugin}
Das SteamVR Plugin bildet wie bereits erwähnt und in Abbildung XX illustriert das Gegenstück zur SteamVR Anwendung und vollendet die Schnittstelle zwischen Unity und der VR Hardware. Des Weiteren bietet das Plugin eine große Menge an vorgefertigten Funktionalitäten, wie z.B. das in Abbildung \ref{fig:CodeDarstellung} angedeutete SteamVRTrackedObject Skript.

\subsubsection{Final IK Plugin für die Unity Engine}\label{sec:FinalIKPlugin}
Ein weiteres wichtiges Plugin ist das Final IK Plugin von dem Entwickler RootMotion. Das IK im Namen des Plugins steht für Inverse Kinematics (deutsch: Inverse Kinematik), also „aus vorhandenen Koordinaten Gelenkwinkel berechnen“ [31, Ham., S.20]. Im Kontext dieser Arbeit bedeutet dies, dass mit Hilfe der gelieferten Koordinaten der Tracker die entsprechenden Gelenkwinkel für die Körperteile berechnet werden. Neben diesen Berechnungen liefert das Final IK Plugin noch das eigentliche Modell des Menschen und einige weitere Funktionalitäten, wie z.B. die Kalibrierung des Modells.

%--------------------------------------------------------------------------------------------------
\section{Das Menschmodell}\label{sec:DasMenschmodell}
In diesem Abschnitt zunächst wird der Aufbau und der Nutzen des Menschmodells ohne die Interaktionsschnittstelle erläutert, da diese beiden Komponenten aufgrund der Anforderungen an die Modularität unabhängig voneinander implementiert wurden.

\subsection{Der Aufbau des Menschmodells}\label{sec:MMAufbau}
Das entstandene Menschmodell basiert auf dem durch das Final IK Plugin gelieferten Modells. Im Folgenden werden die einzelnen Komponenten erläutert. Zu den Komponenten gehören das eigentliche Modell, die Skripte und die ergänzenden Komponenten die durch dieser Arbeit dazugekommen sind.

\subsubsection{Das eigentliche Modell}\label{sec:MMModell}
Das Final IK Plugin liefert mehrere Modelle. Für diese Arbeit wurde das Modell mit dem Namen Dummy ausgewählt, welches wie der Name bereits vermuten lässt wie ein Crashtest-Dummy aussieht (Vgl. Abbildung \ref{fig:Dummy}).
\newline\newline
Das Modell in der Szene trägt den Namen \textbf{Dummy} und enthält das sogenannte \textbf{VR IK Skript} des Final IK Plugins. Dieses Skript kümmert sich um die Animation des Modells, daher sind Referenzen zu allen Körperteilen gesetzt. Zu den Körperteilen Kopf, Hände und Füße gibt es zusätzliche Einstellungsmöglichkeiten. Um den Dummy zu verwenden müssen mindestens der Kopf und die beiden Hände als \textbf{Targets} (Ziele für das Tracking) gesetzt werden. Optional können auch die Füße als Targets gesetzt werden. Falls man die Genauigkeit des Trackings um noch eine Stufe verbessern möchte, kann man sogenannte \textbf{Bend Goals} („Beug-Ziele“) setzen (Vgl. Abbildung \ref{fig:TargetBendGoal}). Es gibt insgesamt fünf solcher Bend Goals. Bei dem Kopf ist das zugehörige Bend Goal der Steißbein-Tracker, bei den Händen die Ellenbogen-Tracker und bei den Füßen die Knie-Tracker. Die Targets und Bend Goals werden von dem Skript TrackerAssignment verwaltet, welches im weiteren Verlauf dieses Kapitels erläutert wird.
\begin{figure}[h]
	\centering
	\includegraphics[width=0.4\linewidth]{Bilder/A36_TargetsBendGoals}
	\caption{Targets und Bend Goal am Beispiel des linken Arms, eigene Abbildung}
	\label{fig:TargetBendGoal}
\end{figure}
\newline
Des Weiteren enthält das Objekt Dummy mit dem VR IK Skript zwei weitere Kind-Objekte (Vgl. Abbildung \ref{fig:Dummy}), den \textbf{BipDummy} und den gleichnamigen \textbf{Dummy}. Letzterer ist nur der Skinned-Mesh-Renderer und kümmert sich darum, dass die Texturen der Körperteile gerendert werden. Dies bietet den Vorteil, dass man leicht durch neue Texturen das aussehen des Dummy verändern kann. Das Objekt BipDummy enthält die Transformationen, also Position, Rotation und Skalierung aller Körperteile. Dieser Modulare Aufbau ermöglicht es einem das Menschmodell in Zukunft durch beliebige neue Modelle auszutauschen, solange die Transformationen für die Körperteile Füße, Knie, Steißbein, Hände, Ellenbogen und Kopf vorhanden sind.
\begin{figure}[h]
	\centering
	\includegraphics[width=0.4\linewidth]{Bilder/A34_DummyAufbau}
	\includegraphics[width=0.338\linewidth]{Bilder/A35_Dummy}
	\caption{Final IK Dummy Modell, eigene Abbildung}
	\label{fig:Dummy}
\end{figure}

\subsubsection{Ergänzende Komponenten}\label{sec:MMKomponenten}
Um das Menschmodell einsatzfähig für die VR Hardware zu machen mussten noch zwei weitere Komponenten zu dem eigentlichen Modell hinzugefügt werden. Diese Komponenten sind das SteamVR CameraRig und die Targets (Ziele) für die optionalen Tracker.
\newline\newline
Das \textbf{SteamVR CameraRig} (Vgl. Abbildung \ref{fig:CameraRig}) ist eine von SteamVR zur Verfügung gestellte Komponente zur Eingrenzung des Spielbereichs und Tracking der Controller (Hier: Hände) und der VR-Brille (Hier: Kopf). Die Komponente besteht aus den drei Kind-Objekten Controller (left), Controler (right) und Camera. Controller (left) und Controller (right) enthalten zusätzlich das 3D-Modell des Controllers als Kind-Objekt, um diesen in der virtuellen Welt anzeigen zu können. 
\newline
Durch diese drei Objekte werden die Bewegungen der beiden Controller und des Kopfes in der virtuellen Welt wiedergespiegelt. Des Weiteren ist es wichtig zu erwähnen, dass das CameraRig für diese Arbeit modifiziert wurde. Den beiden Controllern und der Kamera wurden Kopien der Transformationen der beiden Hände und des Kopfes als Kind-Objekte hinzugefügt. Diese Kopien kamen beim VR IK Skript zum Einsatz um Informationen über die Positionen der Hände und des Kopfes zu erhalten (Vgl. Abbildung \ref{fig:TargetBendGoal}), da diese sich als Kind-Objekte der Controller (Hier: Hände) und der Kamera (Hier: Kopf) entsprechend mitbewegen.
\newline
Abschließend sei noch zu erwähnen, dass die Transformation vom Kopf (Bip002 Head) um ca. -0,113 Einheiten entlang der Y- und Z-Achse gegenüber dem Parent-Objekt Camera verschoben wurde, damit die Kamera sich nicht innerhalb des Kopfes des Modells befindet. Die Transformationen von der linken (Bip002 L Hand) und der rechten (Bip002 R Hand) Hand wurden nicht verschoben.
\begin{figure}[h]
	\centering
	\includegraphics[width=0.25\linewidth]{Bilder/A37_CameraRig}
	\caption{Modifiziertes CameraRig, eigene Abbildung}
	\label{fig:CameraRig}
\end{figure}
\newline
Die zweite wichtige ergänzte Komponente sind die sogenannten \textbf{Targets} (Ziele) (Vgl. Abbildung \ref{fig:Targets}). Da nur sichergestellt ist, dass der Kopf und die Hände durch die VR-Brille und die Controller mit Hilfe des SteamVR CameraRigs verfolgt werden bedarf es zusätzliche Verfolgungsziele für die optionalen Tracker an den Füßen, Knien, Ellenbogen und dem Steißbein.
\newline
Die Targets enthalten momentan noch jeweils ein Kind-Objekt (Vgl. Abbildung \ref{fig:Targets}). Diese Objekte sind lediglich für den Entwicklungsprozess gedacht und können auf Wunsch einfach gelöscht werden, da es sich nur um einfache gefärbte Kugeln handelt, die dem Entwickler anzeigen wo die Tracker sich momentan im Raum befinden. 
\newline
Des Weiteren kommen die Targets genauso wie die Transformationen der Hände und des Kopfes im CameraRig beim VR IK Skript zum Einsatz, um dem Skript Informationen über die Positionen der entsprechenden Körperteile zu liefern. Dabei ist anzumerken, dass die Füße und das Steißbein als eigentliche Targets (Ziele) zum Einsatz kommen, während die Knie und die Ellenbogen als Bend Goals („Beug-Ziele“) verwendet werden (Vgl. Abbildung \ref{fig:TargetBendGoal}).
\begin{figure}[h]
	\centering
	\includegraphics[width=0.25\linewidth]{Bilder/A38_Targets}
	\caption{Targets, eigene Abbildung}
	\label{fig:Targets}
\end{figure}
\newline
Jedes dieser Targets enthält wie bereits in Abbildung \ref{fig:CodeDarstellung} illustriert das Skript \textbf{SteamVR Tracked Object} (Vgl. Abbildung \ref{fig:TrackedObject}). Durch das bereits erwähnte Skript TrackerAssignment wird jedem der Targets der Index des zugehörigen Trackers zugewiesen. Die Standardeinstellung ist „none“, da das Menschmodell auch ohne jegliche dieser optionalen Tracker und sogar nur mit einer beliebigen Teilmenge von ihnen funktioniert.
\begin{figure}[h]
	\centering
	\includegraphics[width=0.45\linewidth]{Bilder/A39_SteamVRTrackedObject}
	\caption{SteamVR Tracked Object Komponente, eigene Abbildung}
	\label{fig:TrackedObject}
\end{figure}

\subsubsection{Ergänzender Code}\label{sec:MMCode}
Für das eigentliche Menschmodell ohne die Interaktionsschnittstelle kommen wie bereits in Abbildung \ref{fig:CodeDarstellung} dargestellt zwei weitere Klassen zum Einsatz. Diese beiden Klassen tragen die Namen CalibrationController und TrackerAssignment.
\newline\newline
Die Klasse \textbf{TrackerAssignment} (Vgl. Abbildung \ref{fig:MenschUML}) kümmert sich wie bereits erwähnt um die Zuweisung der Tracker zu den entsprechenden Körperteilen. Aufgrund dessen enthält diese Klasse in öffentlichen Variablen Verweise auf alle SteamVR Tracked Object Skripte der optionalen Tracker sowie die Transformationen der zugehörigen Körperteile. Des Weiteren enthält diese Klasse in öffentlichen Variablen Verweise auf den Calibration Controller und das VR IK Skript vom Dummy. Außerdem enthält diese Klasse noch einige private Variablen, insbesondere die bereits in Kapitel \ref{sec:Variablendefinition} erwähnten IDs der verschiedenen Tracker.
\newline
Insgesamt enthält diese Klasse zwei Methoden. Die erste Methode trägt den Namen Start und wird wie der Name bereits vermuten lässt einmal bei der Initialisierung ausgeführt und danach nie wieder. In dieser Methode werden lediglich ein paar Variablen die im späteren Verlauf verwendet werden deklariert. Die zweite Methode trägt den Namen AssignTrackers und beinhaltet die gesamte Funktionalität dieser Klasse.
\newline
Zunächst werden mittels einer Schleife alle angeschlossenen Geräte durchlaufen und mit den hinterlegten IDs der Tracker verglichen. Sobald die ID von einem der angeschlossenen Geräte mit einer der hinterlegten IDs übereinstimmt, wird dem SteamVR Tracked Object Skripts des Targets des entsprechenden Körperteils der Index dieses Trackers zugewiesen (Vgl. Abbildung \ref{fig:TrackedObject}).
\newline
Daraufhin werden dem Calibration Controller die Referenzen zu den Transformationen der Füße und des Steißbeins übergeben (Vgl. Abbildung \ref{fig:CodeDarstellung}), falls die entsprechenden Tracker aktiv sind. Es ist anzumerken, dass jede beliebige Teilmenge dieser drei Tracker aktiv sein kann, ohne das Ergebnis der Kalibrierung zu verschlechtern.
\newline
Anschließend werden durch den Verweis auf das VR IK Skript die Bend Goals zugewiesen, falls die Tracker für die Knie und Ellenbogen angeschlossen sind. Dafür muss zusätzlich das Bend Goal Weight (die Gewichtung des Bend Goals) auf den Wert 1 gesetzt werden (Vgl. Abbildung \ref{fig:TargetBendGoal}). Hier ist ebenfalls anzumerken, dass jede beliebige Teilmenge der Tracker aktiv sein kann, ohne das Ergebnis der Darstellung des Menschmodells wesentlich zu verschlechtern. Falls Beispielsweise der Tracker am linken Ellenbogen aktiv ist aber der Tracker am rechten Ellenbogen nicht, wird die Position des rechten Ellenbogens einfach durch das VR IK Skript anhand der Ausrichtung der Hand und des gesamten Körpers automatisch approximiert.
\newline\newline
Die Klasse \textbf{CalibrationController} (Vgl. Abbildung \ref{fig:MenschUML}) arbeitet mit dem bereits vorhandenen VR IK Calibrator des Final IK Plugins. Aufgrund dessen enthält die Klasse in öffentlichen Variablen die Verweise auf das VR IK Skript vom Dummy, auf die Transformationen vom Kopf und den beiden Händen und auf die Klasse TrackerAssignment. Des Weiteren enthält die Klasse die öffentlichen Variablen Settings (Einstellungen) und Data (Daten der Kalibrierung). Schließlich enthält die Klasse noch öffentliche aber noch nicht initialisierte Variablen für die Verweise auf die Transformationen des Steißbeins und der beiden Füße.
\newline
Insgesamt enthält die Klasse drei Methoden, wobei die letzte den Namen LateUpdate trägt und in der Regel nicht verwendet wird. Diese Klasse ist nur für Entwicklungszwecke da und war in einer Demo-Klasse des Final IK Plugins vorhanden. Sie ermöglicht es einem die Kalibrierung des Dummys durch das drücken einer bestimmten Taste auf der Tastatur zu starten. Die erste Methode dieser Klasse trägt den Namen Update und wird einmal pro Frame ausgeführt. Aufgabe dieser Methode ist es, sobald vom Bediener die entsprechende Taste am linken Controller gedrückt wird den Kalibrierungsprozess durch die zweite Methode der Klasse mit dem Namen Calibrate zu starten. Es ist anzumerken, dass der Kalibrierungsprozess beliebig oft ausgeführt und somit nach Bedarf erneuert werden kann.
\newline
In der Methode Calibrate wird zunächst durch den Verweis auf die Klasse TrackerAssignment die Zuweisung der einzelnen Tracker gestartet. Hierbei ist anzumerken, dass wie bereits vorher erwähnt und in Abbildung \ref{fig:CodeDarstellung} dargestellt die noch nicht initialisierten Variablen für die Verweise auf die Transformationen des Steißbeins und der beiden Füße initialisiert werden, falls die entsprechenden Tracker angeschlossen sind. Ansonsten werden für die Kalibrierung lediglich die Positionen der Hände und des Kopfes berücksichtigt. In jedem Fall werden für die Kalibrierung die Positionen der Knie und der Ellenbogen nicht berücksichtigt.
\newline
Schließlich wird der eigentliche Kalibrierungsprozess durch die gleichnamige Methode Calibrate der Klasse VR IK Calibrator gestartet, indem die entsprechenden Variablen an diese Methode übergeben werden. Die vorher angesprochenen Variablen Settings und Data sind dafür da, falls man mit Hilfe der Klasse VR IK Calibrator und deren Methoden Kalibrierungen abspeichern und wieder laden möchte. Diese Funktionalität wurde für diese Arbeit nicht berücksichtigt.
\begin{figure}[h]
	\centering
	\includegraphics[width=0.65\linewidth]{Bilder/A40_MenschUML}
	\caption{UML Diagramm der Klassen TrackerAssignment und CalibrationController, eigene Abbildung}
	\label{fig:MenschUML}
\end{figure}

\subsection{Zusammenfassung des Menschmodells}\label{sec:MMFunktionen}
Das Menschmodell ermöglicht eine Abbildung der Bewegungen des Bedieners auf einen virtuellen Klon. Dabei müssen mindestens die VR Brille und die Controller verwendet werden, um die Positionen der Hände und des Kopfes abbilden zu können. Die Bewegungen der restlichen Körperteile werden in diesem Fall durch das VR IK Skript approximiert. Des Weiteren ermöglicht dieses Menschmodell mit Hilfe der in Abbildung XX dargestellten Befestigungen den Einsatz der HTC VIVE Tracker, um die Bewegungen der Füße, der Knie, der Ellenbogen und des Steißbeins zu berücksichtigen und somit ein genaueres virtuelles Abbild der Bewegungen zu erhalten. Es ist anzumerken, dass nicht jeder dieser Tracker aktiv sein muss und sogar jede beliebige Teilmenge dieser optionalen Tracker aktiv sein kann. Damit wird die in Kapitel XX gestellte Anforderung an die Genauigkeit erfüllt. Dadurch, dass die Abbildung der Bewegungen auf den virtuellen Klon in nahe zu Echtzeit stattfindet, wird die in Kapitel XX gestellte Anforderungen Echtzeit ebenfalls erfüllt. Außerdem ist noch anzumerken, dass durch den Kalibrierungsprozess das Menschmodell durch einen einfachen Knopfdruck an die Körpergröße von dem Bediener angepasst werden kann. Schließlich ist anzumerken, dass aufgrund des Modularen Aufbaus des Menschmodells zukünftige Erweiterungen ermöglicht werden und somit die in Kapitel XX geforderte Modularität erfüllt wird. Da das Modell zusätzlich nicht von seiner virtuellen Umgebung abhängig ist und somit in jeder beliebigen Umgebung eingesetzt werden kann wird auch die Anforderung an die Interoperabilität aus Kapitel XX erfüllt.
%--------------------------------------------------------------------------------------------------
\section{Die Interaktionsschnittstelle}\label{sec:DieInteraktionsschnittstelle}
Da das Menschmodell ohne die Möglichkeit mit der Umgebung zu interagieren nur bedingt nützlich ist, wird in diesem Abschnitt der Aufbau und der Nutzen der Interaktionsschnittstelle erläutert. Die Grundidee der Interaktionsschnittstelle ist es das Menschmodell durch einen Pointer (Zeiger) zu erweitern und somit eine intuitive Interaktion mit der Umgebung zu ermöglichen (Vgl. Abbildung XX).
\begin{figure}[h]
	\centering
	\includegraphics[width=0.65\linewidth]{Bilder/A44_InteraktionsBeispiel}
	\caption{Grundidee der Interaktion, eigene Abbildung}
	\label{fig:InteraktionBeispiel}
\end{figure}

\subsection{Der Aufbau der Interaktionsschnittstelle}\label{sec:AufbauInteraktion}
Im Folgenden zunächst das Grundgerüst der Interaktionsschnittstelle erklärt, bevor die zusätzlichen Funktionalitäten erläutert werden.

\subsubsection{Das Grundgerüst der Interaktionsschnittstelle}\label{sec:GrundgerüstInteraktion}
Das Grundgerüst der Interaktionsschnittstelle besteht lediglich aus dem Interaktionssystem, den Pointern, der Klasse die den Wechsel zwischen den beiden Pointern durchführt und der Klasse die das Menü des angeklickten Objektes aufruft. Es wurden zwei Pointer verwendet, da der eine Pointer genutzt wird um mit Objekten wie z.B. Robotern in der Umgebung zu interagieren, während der andere Pointer dafür da ist um mit Menüs zu interagieren. Es ist anzumerken, dass die Klassen VR Input für das Interaktionssystem und VR Canvas Pointer und VR Physics Pointer für die beiden Pointer auf Anleitungen des Youtube Kanals VR with Andres basieren [32, VRWithAndrew].
\newline\newline
Das \textbf{Interaktionssystem} (Vgl. Abbildung XX) basiert auf dem sogenannten Event System von Unity, welches standardmäßig in jeder Szene vorhanden ist. Mit Hilfe des Event Systems werden in der Regel der Input der Tastatur und der Maus gehandhabt. Das Event System wird für diese Arbeit durch die Klasse \textbf{VR Input} erweitert. Die Klasse enthält die drei öffentlichen Variablen eventCamera, clickButton und clickAction. Die Variable eventCamera erhält einen Verweis auf die Kamera des aktuell genutzten Pointers, während die Variablen clickButton und clickAction lediglich genutzt werden um Input-Taste zu deklarieren.
\newline
Insgesamt enthält die Klasse fünf Methoden. Die erste Methode trägt den Namen Awake, dient lediglich zur Initialisierung und wird automatisch ausgeführt sobald eine der Methoden in der Klasse ausgeführt wird. Die nächsten vier Methoden sind dafür da um den Input der Maus zu überschreiben. Die erste dieser vier Methoden trägt den Namen GetMouseButton und liefert Informationen darüber ob Taste gedrückt und wieder losgelassen wurde. Die Methoden GetMouseButtonDown und GetMouseButtonUp ermöglichen es dem Entwickler separat abzufragen, ob die Taste runtergerückt oder losgelassen wurde. Somit wird durch das Klicken der entsprechenden Taste am rechten Controller das Klicken mit der Maus simuliert. Die letzte Methode trägt den Namen mousePosition und überschreibt die Position der Maus durch den exakten Mittelpunkt des Anzeigebildschirms.
\begin{figure}[h]
	\centering
	\includegraphics[width=0.5\linewidth]{Bilder/A41_EventSystem}
	\caption{Event System mit VR Input Aufbau in der Szene, eigene Abbildung}
	\label{fig:EventSystem}
\end{figure}
\newline
Der aktuell eingesetzte Pointer in wird sobald das Programm ausgeführt wird als Kind-Objekt des rechten Controllers (Vgl. Abbildung XX) instanziiert. Somit bewegt sich der Pointer mit der rechten Hand bzw. mit dem rechten Controller mit. Zu Beginn wird Standardmäßig der Physics Pointer instanziiert, sobald jedoch ein Objekt angeklickt wird und sich somit das Menü dieses Objektes öffnet, wird auf den Canvas Pointer gewechselt.
\newline
Sowohl der \textbf{Canvas Pointer} als auch der \textbf{Physics Pointer} enthalten die Komponenten Camera, Line Renderer und das zugehörige Skript. Der Physics Pointer enthält zusätzlich die Komponente Physics Raycaster (Vgl. Abbildung XX). Mit Hilfe der Komponente Camera lässt sich bestimmen worauf der aktuelle Pointer zeigt. Diese Komponente wird durch das im späteren Verlauf dieses Kapitels vorgestellte Skript Switch Pointers beim Event System hinterlegt (Vgl. Abbildung XX). Der Line Renderer hat lediglich die Aufgabe eine Linie zu zeichnen, damit der Bediener erkennen kann worauf er klickt. Diese Linie beginnt beim Controller in der rechten Hand des Bedieners und ist standardmäßig drei Unity-Einheiten lang. Falls ein Objekt die Linie schneidet, endet diese an dem Schnittpunkt. 
\newline
Grundsätzlich funktionieren die Klassen VR Physics Pointer und VR Canvas Pointer nach dem gleichen Prinzip. Es wird ausgehend vom Ursprung des Pointers ein Raycast („Strahl“) gesendet und überprüft womit dieser interagiert. Der unterschied ist, dass durch die Klasse VR Physics Pointer ein Raycast gesendet wird, der nur mit Objekten die einen Collidern besitzen interagiert (Physics Raycast). Die Klasse VR Canvas Pointer hingegen sendet mit Hilfe des Event Systems einen Raycast, der nur mit Objekten die Bestandteil einer graphischen Benutzeroberfläche sind interagiert. Daher ist die einzige Anforderung an die Umgebung, dass die Objekte mit denen interagiert werden soll einen Collider besitzen und das die Menüs die bereits in Unity vorhandenen Elemente wie z.B. Knöpfe oder Slider für grafische Benutzeroberflächen nutzen.
\begin{figure}[h]
	\centering
	\includegraphics[width=0.5\linewidth]{Bilder/A42_CanvasPointer}
	\includegraphics[width=0.5\linewidth]{Bilder/A43_PhysicsPointer}
	\caption{Canvas Pointer und Physics Pointer Aufbau in der Szene, eigene Abbildung}
	\label{fig:Pointer}
\end{figure}
\newline
Der Wechsel zwischen den Pointern wird durch die Klasse \textbf{Switch Pointers} gehandhabt. Diese enthält in öffentlichen Variablen Verweise auf die Objektvorlagen der beiden Pointer (Vgl. Abbildung XX).
\newline
Zu Beginn wird in der Methode Start wie bereits erwähnt ein Physics Pointer instanziiert, dessen Kamera Komponente als Event Camera beim Event System gesetzt wird (Vgl. Abbildung XX). Die Methode Update ruft die Methode HandlePointerSwitch jede Sekunde mehrfach auf. Sobald ein Objekt angeklickt wird und sich sein Menü öffnet, wird der Pointer mit Hilfe der Methode HandlePointerSwitch ausgetauscht und die Event Camera beim Event System aktualisiert. Des Weiteren werden bei dem Canvas Pointer die Referenzen zum Event System und seinem Input Module gesetzt (Vgl. Abbildung XX). Die Beiden Hilfsmethoden SpawnPhysicsPointer und SpawnCanvasPointer kommen zum Einsatz um den entsprechenden Pointer zu instanziieren.
\begin{figure}[h]
	\centering
	\includegraphics[width=0.5\linewidth]{Bilder/A45_SwitchPointer}
	\caption{Switch Pointers Aufbau in der Szene, eigene Abbildung}
	\label{fig:SwitchPointer}
\end{figure}
\newline
Die letzte Komponente des Grundgerüsts der Interaktionsschnittstelle ist die Klasse \textbf{Object Menu}. Diese Klasse kann zu jedem beliebigen Objekt in der Szene hinzugefügt werden. Die wichtigsten Komponenten dieser Klasse sind die öffentlichen Variablen prefab, thisObject, clickButton und clickAction (Vgl. Abbildung XX). Die Variable prefab enthält den Verweis auf die Objektvorlage des zu öffnenden Menüs und die variable thisObject enthält den Objektverweis auf das Objekt zu dem das Skript gehört. Die Variablen clickButton und clickAction werden lediglich genutzt um die Input-Taste für das Schließen des Menüs zu deklarieren.
\newline
Insgesamt enthält diese Klasse drei Methoden. In der Methode Start werden lediglich ein paar private Hilfsvariablen deklariert und in der Methode Update wird, falls das Objekt angeklickt wird die Methode HandleButtonPress aufgerufen. Diese Methode instanziiert dann eine Instanz des zu öffnenden Menüs. Beim erneuten aufrufen der Methode HandleButtonPress durch das Drücken der entsprechenden Taste am rechten Controller wird das bereits geöffnete Menü wieder geschlossen. Des Weiteren aktualisiert die Methode Update die Position des Menüs mehrmals pro Sekunde, sodass sich das Menü mit der Kopfbewegung des Bediener mitbewegt und stehts gut sichtbar ist.
\begin{figure}[h]
	\centering
	\includegraphics[width=0.5\linewidth]{Bilder/A46_ObjectMenu}
	\caption{Object Menu Aufbau in der Szene, eigene Abbildung}
	\label{fig:ObjectMenu}
\end{figure}		% Interaktion mit der Umgebung
%--------------------------------------------------------------------------------------------------	
%--------------------------------------------------------------------------------------------------
\chapter{Einbinden des Menschmodells und Validierung der Anforderungen}\label{cha:ValidierungDesKonzepts}
In diesem Kapitel wird zunächst die Vorgehensweise für das Einbinden des Menschmodells in ein neues Projekt und das Einfügen neuer Objekte in der Umgebung erklärt. Daraufhin wird erläutert, warum die in Kapitel \ref{sec:AnforderungenKonzept} gestellten Anforderungen an das Menschmodell und an die Interaktionsschnittstelle gewährleistet werden.

%--------------------------------------------------------------------------------------------------
\section{Einbinden des Menschmodells in ein neues Projekt}\label{sec:MenschmodellEinbinden}
Um das Einbinden des Menschmodells in ein beliebiges Projekt einfach zu gestalten, gibt es in diesem Abschnitt eine Erläuterung der Vorgehensweise. Dabei sind die folgenden drei Schritte zu befolgen: Vorbereitung der Entwicklungsumgebung, Einfügen der Vorlage in das neue Projekt und Einfügen in die Szene.

\subsection{Schritt 1: Vorbereitung der Entwicklungsumgebung}
Im ersten Schritt muss die Entwicklungsumgebung vorbereitet werden. Dafür muss zunächst mit Hilfe der in Kapitel XX erläuterten Programme VIVE Wireless und SteamVR eine Verbindung mit der VR Hardware ermöglicht werden. Es ist zu empfehlen, die aktuellsten Versionen dieser Programme zu nutzen. Des Weiteren muss die Unity Entwicklungsumgebung installiert werden. Dabei ist zu beachten, dass diese Arbeit, wie bereits in Kapitel XX erwähnt, auf Grundlage der Version 2019.2.19f1 von Unity entwickelt wurde. Schließlich müssen noch die in Kapitel XX erläuterten Plugins SteamVR und Final IK aus dem Asset Store heruntergeladen und in dem aktuellen Projekt importiert werden.

\subsection{Schritt 2: Einfügen der Vorlagen in das neue Projekt}
Im zweiten Schritt müssen die Vorlagen des Menschmodells in das neue Projekt eingefügt werden. Es ist ausreichend, die in Kapitel \ref{fig:UnityOverview} dargestellten Ordner 'Prefab', 'Scenes' und 'Scripts' zu kopieren und in das Verzeichnis des neuen Projekts einzufügen. Dabei ist anzumerken, dass ein Prefab in Unity eine Objektvorlage darstellt, die beliebig oft instanziiert werden kann, daher befinden sich im Ordner 'Prefab' sämtliche Objektvorlagen des Projekts. In dem Ordner 'Scenes' befinden sich drei Demo-Szenen, unter anderem auch die in Abbildung \ref{fig:UnityOverview} dargestellte Szene und im Ordner 'Scripts' befinden sich sämtliche Skripte des Projekts. Eine genauere Erläuterung der Ordnerstruktur folgt im weiteren Verlauf dieses Kapitels.

\subsection{Schritt 3: Einfügen in die Szene}
Im letzten Schritt muss lediglich das Menschmodell in die Szene eingefügt werden. Dabei steht dem Entwickler frei, ob das Menschmodell, die Interaktionsschnittstelle oder beides zusammen in das neue Projekt eingebunden werden soll. Um das Menschmodell in das neue Projekt einzubinden, muss lediglich das Prefab 'Human' eingefügt werden, während bei der Interaktionsschnittstelle neben dem Prefab 'InteractionEventSystemAndScripts' noch das SteamVR CameraRig in der Szene eingefügt werden muss. Falls das Menschmodell in Kombination mit der Interaktionsschnittstelle verwendet werden soll, ist es ausreichend wie in Abbildung \ref{fig:UnityOverview} dargestellt, einfach das Prefab 'Human+Interaction' in der Szene zu platzieren.

\section{Vorgehensweise beim hinzufügen neuer Objekte in die Szene}\label{sec:ObjekteEinbinden}
Genauso wie das Menschmodell inklusive der Interaktionsschnittstelle können beliebige Objekte ohne großen Aufwand in das Unity Projekt integriert werden. Dabei sind die folgenden drei Schritte zu befolgen: Einfügen in die Ordnerstruktur des Projekts, Einfügen in die Szene und Einbinden in die Menüs.

\subsection{Schritt 1: Einfügen in die Ordnerstruktur des Projekts}
\begin{figure}[h]
	\centering
	\includegraphics[width=1\linewidth]{Bilder/A54_Ordnerstruktur}
	\caption{Die Ordnerstruktur, eigene Abbildung}
	\label{fig:Ordnerstruktur}
\end{figure}
\noindent Bevor erklärt wird, wie neue Objekte in das Projekt eingefügt werden, gibt es zunächst eine Einführung in den Aufbau der Ordnerstruktur des Projekts. In der Abbildung \ref{fig:Ordnerstruktur} sind die wichtigsten Ordner des Projekts abgebildet. Es ist anzumerken, dass Ordner durch Rechtecke und Objekte durch Ovale abgebildet werden.
\newline\newline
Auf der obersten Ebene befinden sich die Ordner \textbf{'Prefab'}, \textbf{'Scenes'} und \textbf{'Scripts'}. In dem Ordner 'Scenes' sind wie bereits erwähnt die drei Demo-Szenen und in dem Ordner 'Scripts' die Skripte abgespeichert. Neben den vorhandenen Demo-Szenen empfiehlt es sich in Zukunft alle weiteren Szenen in diesem Ordner abzuspeichern. Im Gegensatz dazu sollte der Inhalt des Ordners 'Scripts' größtenteils unverändert bleiben, da sich in diesem Ordner lediglich die Skripte für das Menschmodell und die Interaktionsschnittstelle verändern. Lediglich die Skripte Spawning Handler und Global Variables, die sich in einem Unterordner befinden, sollten angepasst werden, falls neue Objekte in die Umgebung eingefügt werden. Des Weiteren ist anzumerken, dass sich neben diesen drei Ordnern noch weitere Ordner auf dieser Ebene im Projekt befinden. Diese sind aber, wie z.B. in Abbildung \ref{fig:UnityOverview} zu erkennen ist, automatisch generierte Ordner die durch das Importieren der Plugins Final IK und SteamVR entstehen. 
\newline\newline
Wie bereits erwähnt, ist der Inhalt der Ordner 'Scenes' und 'Scripts' bereits bekannt. Aufgrund dessen sind in Abbildungen \ref{fig:Ordnerstruktur} des Ordners 'Prefab' abgebildet.
--> TODO erklären

\subsection{Schritt 2: Einfügen in die Szene}
--> PointerEvent Object + Collider + Platzieren

\subsection{Schritt 3: Einbinden in die Menüs}
--> Wie bereits erwähnt das mit den 2 skripten
--> Es empfiehlt sich diese neu zu schreiben, da diese wie bereits erwähnt nur zu demo zwecken erstellt wurden
--> Auch das mit falko hier

%--------------------------------------------------------------------------------------------------
\section{Validierung des Anforderungen des Menschmodells}\label{sec:ValidMensch}

\subsection{Anforderung 1: Genauigkeit}
Durch die Möglichkeit, die Bewegungen des Bedieners an bis zu zehn Körperteilen (Beide Füße, beide Knie, Steißbein, beide Hände, beide Ellenbogen und Kopf) zu verfolgen und diese Daten bei der Abbildung auf den virtuellen Menschen zu berücksichtigen, wird die in Kapitel \ref{sec:AnforderungenKonzept} geforderte Genauigkeit erfüllt. Zusätzlich wird die Genauigkeit, durch die Möglichkeit das Menschmodell zu Kalibrieren, also an die Körpergröße des Bedieners anzupassen, verbessert. So könnte man in Zukunft für jeden beliebigen Bediener mit Hilfe des Menschmodells einen virtuellen Klon für die virtuelle Welt schaffen, in dem man die Textur für den in Kapitel \ref{sec:MMModell} angesprochenen Skinned Mesh Renderer anpasst.

\subsection{Anforderung 2: Echtzeit}
Mit Hilfe des Plugins Final IK werden die Bewegungsdaten des Bedieners in nahezu Echtzeit verarbeitet und auf das Menschmodell übertragen. Es sind keine sichtbaren Verzögerungen zu erkennen, die die Nützlichkeit des Menschmodells einschränken würden oder sogar eine potenzielle Gefahrenquelle darstellen könnten. Aufgrund dessen wird die in Kapitel \ref{sec:AnforderungenKonzept} gestellte Anforderung an die Echtzeit ebenfalls erfüllt.

\subsection{Anforderung 3: Interoperabilität}
Das Menschmodell wurde umgebungsunabhängig implementiert, ist also von keinen anderen Komponenten der virtuellen Umgebung abhängig. Folglich kann das Menschmodell in jeder beliebigen virtuellen Umgebung, wie Beispielsweise virtuell begehbare Produktionsanlagen, eingesetzt werden und erfüllt somit die in Kapitel \ref{sec:AnforderungenKonzept} gestellte Anforderung an die Interoperabilität.

\subsection{Anforderung 4: Modularität}
Wie in Abbildung \ref{fig:UnityOverview} zu erkennen ist, ist das Menschmodell modular aufgebaut und erlaubt einfache Anpassungen und Erweiterungen in der Zukunft. Insgesamt besteht das Menschmodell (ohne die Interaktionsschnittstelle) aus den vier Komponenten Kamera, Modell, Skripte und Verfolgungsziele, welche selber nochmal aus einigen Komponenten bestehen. Aufgrund dessen wird die in Kapitel \ref{sec:AnforderungenKonzept} geforderte Modularität gewährleistet.

%--------------------------------------------------------------------------------------------------
\section{Validierung der Anforderungen der Interaktionsschnittstelle}\label{sec:ValidInteraktion}

\subsection{Anforderung 1: Bidirektionalität}
Der Informationsaustausch zwischen Mensch und Maschine findet bidirektional statt, da der Mensch durch den Pointer die Möglichkeit erhält über graphische Benutzeroberflächen mit der Maschine zu interagieren. In anderen Worten stellt der Pointer das Input-Medium des Menschen dar. Gleichzeitig ermöglichen die graphischen Benutzeroberflächen die Darstellung von Feedback der Produktionsanlagen. So könnten Beispielsweise Produktionsraten, Stromverbrauch oder sonstige produktionstechnisch relevante Parameter angezeigt werden. Aufgrund dessen wird die in Kapitel \ref{sec:AnforderungenKonzept} geforderte Anforderung der Bidirektionalität erfüllt.

\subsection{Anforderung 2: Genauigkeit}
Mit Hilfe des Pointers, der durch das Bewegen der rechten Hand gesteuert wird, wird ein sehr präzises und vor allem intuitives interagieren mit der Umgebung ermöglicht. Aufgrund dessen wird die in Kapitel \ref{sec:AnforderungenKonzept} geforderte Genauigkeit bei der Interaktion mit der Umgebung gewährleistet.

\subsection{Anforderung 3: Echtzeit}
Durch die Verarbeitung der Bewegungsdaten in Echtzeit wird nicht nur die nahezu verzögerungslose Abbildung des Menschmodells, sondern auch eine nahezu verzögerungslose Interaktion mit der Umgebung ermöglicht. Dies ermöglicht den Bedienern schnell auf Veränderungen in der virtuellen Umgebung zu reagieren und spontane Anpassungen zu tätigen. Aufgrund dessen wird die in Kapitel \ref{sec:AnforderungenKonzept} gestellte Anforderung an die Echtzeit ebenfalls erfüllt.

\subsection{Anforderung 4: Interoperabilität}
Die Interaktionsschnittstelle ist, bis auf wenige Skripte und dem Interaktionssystem, nicht von der Umgebung abhängig. Daher ist es ausreichend, die eben angesprochenen Komponenten irgendwo in der Szene zu hinterlegen (Vgl. Abbildung \ref{fig:UnityOverview}). Um eine Interaktion mit beliebigen Objekten in der Szene zu ermöglichen, müssen diese lediglich mit dem Object Menu Skript und einem Collider erweitert werden. Des Weiteren müssen ihre Menüs mit Hilfe der bereits in Unity vorhanden Komponenten für graphische Benutzeroberflächen implementiert sein. Folglich wird die in Kapitel \ref{sec:AnforderungenKonzept} geforderte Interoperabilität gewährleistet, da sich die beschrieben Funktionalität auf beliebige Objekte übertragen lässt, solange die entsprechenden Rahmenbedingungen eingehalten werden.

\subsection{Anforderung 5: Modularität}
Sowohl das Menschmodell, als auch die Interaktionsschnittstelle sind Modular aufgebaut und bestehe aus austauschbaren Komponenten. Es ist beispielsweise möglich, die Interaktionsschnittstelle für andere VR Hardware einsatzfähig zu machen. Dafür müssen lediglich die entsprechenden Input Schnittstellen der VR Hardware in einigen Skripten angepasst werden. Des Weiteren wurden das Menschmodell und die Interaktionsschnittstelle getrennt voneinander entwickelt, um die Unabhängigkeit und somit die in Kapitel \ref{sec:AnforderungenKonzept} geforderte Modularität zu gewährleisten.

%--------------------------------------------------------------------------------------------------			% Validierung des Konzepts
%--------------------------------------------------------------------------------------------------	
%--------------------------------------------------------------------------------------------------
\chapter{Ausblick und Fazit}\label{cha:AusblickUndFazit}

Einige der in Kapitel \ref{sec:PotentialeIndustrie4.0} dieser Arbeit vorgestellten Potenziale von Industrie 4.0, konnten durch das bei dieser Arbeit entstandene und in Kapitel \ref{cha:Umsetzung} vorgestellte Anwendungsbeispiel einer virtuellen Umgebung, in der mittels eines Menschmodells und einer Interaktionsschnittstelle mit beliebigen Objekten interagiert werden kann, verdeutlicht werden. Dabei sind die Potenziale zehn („Simulation und Überwachung der Produktion") und elf („Chancen für IT-Unternehmen") besonders hervorzuheben.
\newline
Mit Hilfe der in Kapitel \ref{sec:PhysischZumKlon} vorgestellten Vorgehensweise für die Schaffung eines virtuellen Klons einer Produktionsanlagen und mit Hilfe des in Kapitel \ref{cha:Umsetzung} vorgestellten Menschmodells inklusive der Interaktionsschnittstelle, eröffnet sich die Möglichkeit, in Zukunft Produktionsstätten über einen virtuellen Zwilling zu begehen, mit einzelnen Produktionsanlagen zu interagieren und diese somit zu steuern. Aufgrund dessen könnte in Zukunft die Wettbewerbsfähigkeit des Hochlohnstandorts Deutschland verbessert werden, da die über einen virtuellen Zwilling gesteuerten Produktionsanlagen an jedem beliebigen Standort auf der Welt stehen könnten.
\newline
Angetrieben durch den damit verbundenen technologischen Entwicklungsaufwand, eröffnen sich vor allem für IT-Unternehmen Chancen am Wandel zu Industrie 4.0 zu profitieren, da der Wandel zu Industrie 4.0 ein von vielen Herausforderungen geprägter und interdisziplinärer Prozess ist und die daher Ausmaße des klassischen Maschinenbaus übersteigt. Dies wird unter anderem durch die in Kapitel \ref{sec:HerausforderungenUmsetzung} vorgestellten Herausforderungen verdeutlicht. Insbesondere fehlt der zweite Schritt der in Kapitel \ref{sec:MMInteraktion} erläuterten Mensch-Maschine-Interaktion, um die in Unity erschaffene, begehbare und interagierbare virtuelle Umgebung mit den realen Produktionsanlagen zu verbinden. Diese Aufgabe war zum Zeitpunkt des Schreibens dieser Arbeit ebenfalls das Thema einer Bachelorarbeit eines Informatik Studenten am Fachgebiet DiK (Datenverarbeitung in der Konstruktion) der Technischen Universität Darmstadt.

%--------------------------------------------------------------------------------------------------
\section{Ausblick}\label{sec:Ausblick}
Es bleibt die Frage, wie Industrie 4.0 in Zukunft aussehen wird. Durch die gewonnenen Erkenntnisse beim Recherchieren und Verfassen dieser Arbeit, lässt sich eindeutig sagen, dass es keine allgemeine Lösung gibt und geben kann. Wie bereits in Kapitel \ref{sec:LeitfadenUmsetung} angedeutet, stehen Unternehmen vor der schwierigen Herausforderungen, die Industrie 4.0 Potenziale in ihrem Unternehmen zu erkennen und unter Berücksichtigung der in Kapitel \ref{sec:HerausforderungenUmsetzung} vorgestellten Herausforderungen umzusetzen. Diese Potenziale können im Zeitalter des Internets der Dinge und Dienstleistungen nicht nur im optimieren der Produktionsprozesse oder der Ressourceneffizienz, sondern auch in der Gestaltung neuer Dienstleistungen (Smart Services) und vielen weiteren Bereichen liegen (Vgl. Kapitel \ref{sec:PotentialeIndustrie4.0}). 
Einziger Anhaltspunkt für Unternehmen sind die Industrie 4.0 Leitfäden aus Kapitel \ref{sec:LeitfadenUmsetung}. Dabei ist anzumerken, dass diese mit Hilfe von Werkzeugkästen, Handlungsfeldern, Rahmenbedingungen oder ähnlichem die Unternehmen lediglich dabei Unterstützen sollen, einen eigenen Weg für die Umsetzung von Industrie 4.0 zu finden.
\newline\newline
Des Weiteren stellt sich die Frage, inwiefern Virtual Reality (virtuelle Realität) beim Wandel zu Industrie 4.0 eine Rolle spielt. Wie bereits in Kapitel \ref{sec:VRGeschichte} dargestellt, befinden wir uns am Anfang einer neuen Phase der Virtual Reality. Dabei findet Virtual Reality neue Einsatzmöglichkeiten in verschiedensten Lebensbereichen, von Bildung bis hin zu Industrie. Als Teil der Disziplin Visual Computing, welche als eine Schlüsselkomponente beim Wandel zu Industrie 4.0 angesehen wird \cite[S.1]{17}, eröffnen sich durch den Einsatz von Virtual Reality viele Potenziale für neue Wertschöpfungsprozesse. Dazu gehört Beispielsweise, die bereits mehrfach erwähnte Möglichkeit einer standortunabhängigen Steuerung von Produktionsanlagen.

%--------------------------------------------------------------------------------------------------3
\section{Zusammenfassung der wichtigsten Ergebnisse}\label{sec:ZusammenfassungErgebnisse}
Zunächst wurde in Kapitel \ref{cha:StandDerTechnik} dieser Arbeit der aktuelle Stand der Technik, insbesondere im Hinblick auf Industrie 4.0, der Ausgangslage der industriellen Fertigung und der Geschichte von Virtual Reality aufgearbeitet
Daraufhin wurde in Kapitel \ref{cha:AufbauDesKonzepts}, basierend auf dem Prozess von der physischen Produktionsanlage bis hin zu ihrem virtuellen Klon, der Mensch-Maschine Interaktion im Kontext dieser Arbeit und dem FMI Standard, ein Konzept geschaffen und Anforderungen definiert.
Anschließend wurde in Kapitel \ref{cha:Umsetzung}, nach einer Einführung in die verwendete Hardware und Software, die Umsetzung des Menschmodells und der Interaktionsschnittstelle mit Hilfe der Unity Engine erläutert. 
Schließlich wurde in Kapitel \ref{cha:ValidierungDesKonzepts} die Vorgehensweise beim Einbinden des entstandenen Menschmodells in ein neues Projekt und die Vorgehensweise beim Einfügen neuer Objekte in der Umgebung erläutert, bevor die zuvor gestellten Anforderungen an das Menschmodell und die Interaktionsschnittstelle validiert wurden.
\newline\newline
Die zentrale Aufgabe dieser Arbeit war es, ein Menschmodell zu schaffen, welches die Bewegungen des Bedieners in der virtuellen Welt verzögerungsfrei Abbildet und dem Bediener ermöglicht mit der virtuellen Umgebung zu interagieren. Dabei mussten insbesondere die letzten zwei, der in Kapitel \ref{sec:AnforderungenKonzept} erläuterten Anforderungen Genauigkeit, Echtzeit, Interoperabilität und Modularität gewährleistet werden, um zukünftige Anpassungen am Menschmodell und das problemlose Zusammenfügen des eigentlichen Menschmodells mit der separat entwickelten Interaktionsschnittstelle zu ermöglichen.
\newline
Wie bereits erwähnt, wurde die Interaktionsschnittstelle unabhängig vom Menschmodell, mit Fokus auf die in Kapitel \ref{sec:AnforderungenKonzept} gestellten Anforderungen Bidirektionalität, Genauigkeit, Echtzeit, Interoperabilität und Modularität, entwickelt. Sowohl die Anforderungen an das Menschmodell, als auch die Anforderungen an die Interaktionsschnittstelle konnten durch die Art und Weise der Umsetzung, wie bereits in Kapitel \ref{sec:ValidMensch} und \ref{sec:ValidInteraktion} erläutert, gewährleistet werden.
\newline
Wegweisend waren dabei die in Kapitel \ref{sec:DasFMU} gewonnenen Kenntnisse über das Functional-Mockup Interface, da das entstandene Menschmodell, ähnlich wie eine FMU bei einer Co-Simulation, nur ein Bestandteil einer größeren Anwendung ist. Die Vollendung der Mensch-Maschine Schnittstelle, also das Verbinden der virtuellen Umgebung in Unity mit echten Produktionsanlagen, durch das Zusammenfügen mit der bereits angesprochenen Arbeit eines anderen Studenten am DiK, sollte dank der Einhaltung der Anforderungen an die Modularität und Interoperabilität, mit geringen Aufwand möglich sein.
\newline\newline
Aufgrund des entstandenen Anwendungsbeispiels, vor allem in Kombination mit dem zweiten Schritt der Mensch-Maschine Interaktion, offenbart sich die Möglichkeit einer maßgeblichen Wandlung der Industrie, bei der insbesondere die Standortunabhängigkeit von Unternehmen gestärkt wird. Dadurch könnte in Zukunft das zentrale Leitmotiv von Firmen aus dem Produktionsbereich "Designed by us, produced anywhere" [Yübo Wang, M.Sc. M.A.] lauten.
%1 \cite{1} 2 \cite{2} 3 \cite{3} 4 \cite{4} 5 \cite{5} 6 \cite{6} 7 \cite{7} 8 \cite{8}
%9 \cite{9} 10 \cite{10} 11 \cite{11} 12 \cite{12} 13 \cite{13} 14 \cite{14} 15 \cite{15}
%16 \cite{16} 17 \cite{17} 18 \cite{18} 19 \cite{19} 20 \cite{20} 21 \cite{21} 22 \cite{22}
%23 \cite{23} 24 \cite{24} 25 \cite{25} 26 \cite{26} 27 \cite{27} 28 \cite{28} 29 \cite{29}
%30 \cite{30} 31 \cite{31} 32 \cite{32} 33 \cite{33} A1 \cite{A1} A2 \cite{A2} A14 \cite{A14}
%A15 \cite{A15} A16 \cite{A16} A18 \cite{A18} A26 \cite{A26} A27 \cite{A27} A28 \cite{A28}
%A29 \cite{A29}
%--------------------------------------------------------------------------------------------------					% Ausblick und Fazit
%--------------------------------------------------------------------------------------------------	
\cleardoublepage
\pagenumbering{roman}\setcounter{page}{6}
\phantomsection
\addcontentsline{toc}{chapter}{Literaturverzeichnis}
\printbibliography											% Literatur Verzeichnis
\cleardoublepage
%--------------------------------------------------------------------------------------------------	
\phantomsection
\addcontentsline{toc}{chapter}{Anhang}
\chapter*{Anhang} \label{cha:Anhang}						% Anhang
\cleardoublepage
%--------------------------------------------------------------------------------------------------	
\end{document}												% Ende des Dokuments