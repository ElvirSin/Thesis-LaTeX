%--------------------------------------------------------------------------------------------------
\chapter{Aufbau des Konzepts}\label{cha:AufbauDesKonzepts}

In diesem Kapitel wird der Aufbau des Konzepts erklärt. Daher gibt es zu erst einen Einblick in den Digitalisierungsprozess von Produktionseinlagen. Daraufhin wird diese Arbeit in das Gesamtkonzept der Mensch-Maschine Interaktion eingeordnet. Schließlich wird der Aufbau des Modells mithilfe des Functional-Mockup Interfaces erklärt.

%--------------------------------------------------------------------------------------------------
\section{Von der physischen Produktionsanlage zum virtuellen Klon}\label{sec:PhysischZumKlon}

Um die Produktionsabläufe einer Fabrik in einer virtuellen Welt abzubilden gibt es insgesamt sieben Schritte die befolgt werden müssen.

\begin{enumerate}
	\item \textbf{Scannen der Fabrik} \\
	Im ersten Schritt werden die Produktionsräume inklusive der Produktionsanlagen gescannt. Das Ergebnis dieses Scans ist eine Punktwolke.
	\item \textbf{Umwandlung des Scans} \\
	Der Scan muss von einer Punktewolke in CAD Modelle umgewandelt werden. Da die CAD Modelle zu groß und daher unpraktikabel für die Entwicklungsumgebung Unity sind, werden diese in das OBJ Format umgewandelt. Die Modelle sind für den späteren Aufbau des virtuellen Klons der echten Produktionsanlage notwendig.
	\item \textbf{Abbildung: Produktionsanlage $\rightarrow$ Modell} \\
	Es wird ein physisches Modell zur Darstellung einiger wichtiger Herausforderungen der echten Produktionsanlage gebaut. Dies könnte Beispielsweise die Form einer Produktionsstraße mit mehreren Stationen besitzen.
	\item \textbf{Virtueller Aufbau} \\
	Das physische Modell der echten Produktionsanlage aus Schritt drei wird in der virtuellen Welt abgebildet (virtueller Klon) und mit dem Menschmodell begehbar gemacht. Dieser Schritt erlaubt die virtuelle Planung einer Produktionsanlage.
	\item \textbf{Konnektivität und Kommunikation} \\
	Die Produktionsanlagen des physischen Modells werden mit der Cloud vernetzt um eine bidirektionale Kommunikation zwischen dem physischem Modell und dem virtuellen Klon zu ermöglichen.
	\item \textbf{Interaktion} \\
	Die Interaktion mit zwischen dem Bediener und den Produktionsanlagen in der virtuellen Welt wird implementiert. Es müssen unter Umständen anwendungsspezifische Benutzeroberflächen implementiert werden.
	\item \textbf{Skalierung} \\
	Skalierung der Vorgehensweise durch Anwendung von Schritt vier bis sechs auf die echte Produktionsanlage.
\end{enumerate}
Durch das Einhalten dieser Schritte erhält man einen virtuellen Klon der echten Produktionsanlage. Diesen virtuellen Klon kann man für viele Zwecke einsetzen, dazu gehören Beispielsweise die Produktionsplanung oder das Fernsteuern von Produktionsanlagen. Das Ergebnis dieser Arbeit kommt bei Schritt vier und sechs zum Einsatz [Quelle Yübo?].

%--------------------------------------------------------------------------------------------------
\section{Mensch-Maschine Interaktion im Kontext dieser Arbeit}\label{sec:MMInteraktion}
\begin{figure}[h]
	\centering
	\includegraphics[width=0.7\linewidth]{Bilder/A19_MMI}
	\caption{Mensch-Maschine Interaktion im Kontext dieser Arbeit, eigene Abbildung}
	\label{fig:MMI}
\end{figure}
\noindent Durch die Abbildung der echten Produktionsanlage in der virtuellen Welt findet die Mensch-Maschine-Interaktion in zwei Schritten statt. Im ersten Schritt interagiert der Mensch mit der Software (dem Computer) über VR-Hardware. Dieser Computer interagiert dann über eine Cloud mit der echten Produktionsanlage. Die Kommunikation findet in beiden Schritten bidirektional statt.
\newline\newline
Diese Arbeit befasst sich mit dem ersten Schritt der oben erklärten Mensch-Maschine Interaktion. Für dem Rest dieser Arbeit werden wir annehmen, dass es bereits eine geeignete Infrastruktur für die Kommunikation zwischen dem Computer und der Produktionsanlage über eine Cloud gibt.

%--------------------------------------------------------------------------------------------------
\section{Das Modell der Interaktivität mit einem Menschmodell}\label{sec:ModellAufbau}
Basierend auf den im folgenden definierten Anforderungen soll mit Hilfe des Functional-Mockup Interfaces ein Modell aufgebaut werden.

\subsection{Anforderungen an das Modell}\label{sec:AnforderungenModell}
Da das Modell aus zwei Teilen besteht, haben beide Teile Ihre eigenen Anforderungen. Diese werden im Folgenden erläutert. Das Ziel ist es ein Menschmodell zu schaffen, welches interaktiv mit seiner Umgebung interagieren kann.

\subsubsection{Menschmodell}
\begin{enumerate}
	\item \textbf{Echtzeit} \\
	Sowohl das Tracking des Bedieners als auch die Abbildung der Tracking-Daten auf das virtuelle menschliche Abbild sind echtzeitkritische Anwendungen. Die Verzögerungen müssen möglichst gering sein um eine gute Nutzbarkeit zu ermöglichen.
	\item \textbf{Interoperabilität} \\
	Das Menschmodell muss umgebungsunabhängig implementiert werden. Die bedeutet, dass das Menschmodell in beliebigen Umgebungen (Fabriken, Produktionshallen, etc.) eingesetzt werden kann.
	\item \textbf{Modularität} \\
	Sowohl einzelne Komponenten als auch das ganze Menschmodell an sich unterliegen der Anforderung der Modularität, um in Zukunft Verbesserungen oder Erweiterungen am Modell durchführen zu können.
\end{enumerate}

\subsubsection{Interaktion}
\begin{enumerate}
	\item \textbf{Bidirektional} \\
	Der Informationsaustausch muss bidirektional stattfinden. Das bedeutet, dass der Bediener über geeignete, leicht zu verstehende, optisch ansprechende und vor allem einheitliche Benutzeroberflächen interagieren kann. Einerseits ermöglichen diese Benutzeroberflächen dem Bediener das Interagieren mit Produktionsanlagen, andererseits werden auf diesen Benutzeroberflächen für den Bediener relevante Informationen in geeigneter Form dargestellt.
	\item \textbf{Interoperabilität} \\
	Die Interaktion muss über eine einheitliche Schnittstelle stattfinden um Interoperabilität zu gewährleisten, sodass sich die Funktionalität auf beliebige Maschinen und Produktionsanlagen übertragen lässt.
	\item \textbf{Modularität} \\
	Durch das Gewährleisten der Modularität wird sichergestellt, dass die Interaktion in Zukunft auch auf neue VR-Hardware übertragen werden kann ohne das die gesamte Software neu geschrieben werden muss.b
\end{enumerate}
Durch das einhalten dieser Herausforderungen ergibt sich ein Menschmodell, welches in Echtzeit die Bewegungen des Bedieners auf das virtuelle Menschliche Abbild projiziert. Des Weiteren ermöglicht das Modell dem Bediener in der virtuellen Welt mit dem virtuellen Klon der Produktionsanlage zu interagieren. Da von Anfang an ein Fokus auf Interoperabilität und Modularität gelegt wird, ist das ganze Modell erweiterbar und an verschiedene Hardware und unterschiedliche Einsatzzwecke anpassbar.

\subsection{Das Functional-Mockup Interface}\label{sec:DasFMU}
Das Functional-Mockup Interface (FMI) ist ein ursprünglich von der Daimler AG entwickelter unabhängiger Standard für den Austausch von dynamischen Modellen und Co-Simulation. Zum Einsatz kommt das Functional-Mockup Interface beispielsweise beim Modellaustausch oder der Co-Simulation zwischen Zulieferern und den OEMs (Original Equipment Manufacturer) [24, FMI-Paper, S.1].
\newline
Anfang 2010 wurde die erste Version (Version 1.0) des FMI Standards veröffentlicht, welche zunächst nur den Modellaustausch unterstützte. Einige Monate später folgte dann die Unterstützung für Co-Simulation. Mittlerweile gab weitere Nachfolger Versionen. Die aktuellste Version (Version 2.0.1) wurde im Oktober 2019 veröffentlicht, unterscheidet sich aber von der Vorgängerversion (Version 2.0) aus dem Jahr 2014 nur durch einige Bugfixes [2, FMI-Spez, S.2].
\newline
Da es neben den anwendungsspezifischen Standards (z.B. Matlab/Simulink: S-Functions) zunächst keinen unabhängigen Standard für den Austausch von Modellen oder die Co-Simulation gab [24, FMI-Paper, S.1], liegt der größte Vorteil des FMI-Standards in seiner Unabhängigkeit.
\newline\newline
Wie bereits erwähnt gibt es für den FMI Standard zwei übergeordnete Anwendungsszenarien:
\begin{enumerate}
	\item \textbf{Modellaustausch} \\
	Der FMI Standard ermöglicht Modellierungsumgebungen ein dynamisches System Modell in C-Code zu repräsentieren, sodass dieses Modell auch in anderen Modellierungsumgebungen genutzt werden kann. Ermöglicht wird dies durch die Darstellung der Modelle mit Hilfe von mathematischen Gleichungen (z.B. Algebraische Gleichungen, Differentialgleichungen oder Diskrete Gleichungen), welche auch Zeit-, Zustands- oder Schritt-Events enthalten können. Es werden große Modelle und sowohl die offline als auch die online Simulation unterstützt [25, FMI-Spez, S.4].
	\item \textbf{Co-Simulation} \\
	Des Weiteren ermöglicht der FMI Standard das Verbinden von mehreren Subsystem, welche über vordefinierte Kommunikationsschnittstellen kommunizieren und somit eine Co-Simulationsumgebung schaffen. Dabei wird der Datenaustausch und die Synchronisation zwischen den einzelnen Subsystemen durch einen sogenannten Master Algorithmus gesteuert. Bei den Master Algorithmen ist zu beachten, dass diese kein Teil des eigentlichen FMI Standards sind [25, FMI-Spez, S.4]. Daher sind die Master Algorithmen nicht standardisiert und können unterschiedliche zusätzliche Funktionalitäten wie z.B. den Austausch von Variablen unterstützen [24, FMI-Paper, S.1].
\end{enumerate}
\noindent Text text text
\begin{figure}[h]
	\centering
	\includegraphics[width=0.7\linewidth]{Bilder/A20_FMIOverview}
	\caption{Aufbau der FMI Variablenbeschreibung [25, FMI-Spez, S.30]}
	\label{fig:FMIOverview}
\end{figure}
\newline
\noindent Text text text
\begin{figure}[h]
	\centering
	\includegraphics[width=0.7\linewidth]{Bilder/A21_FMIBlock}
	\caption{FMU Instanz [25, FMI-Spez, S.9]}
	\label{fig:FMIBlock}
\end{figure}
\newline
\noindent Text text text

\subsection{Der Aufbau des Models}\label{sec:ModelAufbau}
%--------------------------------------------------------------------------------------------------