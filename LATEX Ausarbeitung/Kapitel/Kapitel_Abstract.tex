%--------------------------------------------------------------------------------------------------
\chapter*{Kurzfassung}\label{cha:Kurzfassung}

Das Ziel dieser Forschungsarbeit ist es zu untersuchen, inwiefern Virtual Reality Technologien die Realisierung von Industrie 4.0 vorantreiben können. Aufgrund dessen wurde untersucht, welche Potenziale ein, durch die Bewegungen des Bedieners steuerbares, virtuelles menschliches Abbild (Menschmodell), in Kombination mit einer geeigneten Interaktionsschnittstelle für die virtuelle Welt, im Kontext von Industrie 4.0, liefert.
Um diese Forschungsfrage zu beantworten, wurden zunächst, nach der Untersuchung von dem aktuellen Stand der Technik der Industrie, der industriellen Fertigung und der Virtual Reality, in einer Konzeptionsphase Anforderungen aufgestellt, die daraufhin bei der Implementierung eines Anwendungsbeispiels berücksichtigt wurden. Für die Implementierung wurde die Entwicklungsumgebung Unity Engine verwendet. Daraufhin wurden die aufgestellten Anforderungen anhand der fertigen Implementierung validiert.
Das entstandene Anwendungsbeispiel ist in der Lage die Potenziale eines solchen Menschmodells in der Industrie aufzuzeigen. Durch das Menschmodell könnte beispielsweise die standortunabhängige Begehung und Steuerung von Produktionsanlagen, über einen entsprechenden virtuellen Klon der Anlage, möglich gemacht werden.
Aufgrund dessen lässt sich deutlich sagen, dass Virtual Reality Technologien im Kontext von Industrie 4.0 berücksichtigt werden sollten. Dennoch Bedarf es noch an viel Forschung und Entwicklung in diesem Bereich, da die Wandlung zu Industrie 4.0 einen interdisziplinären Entwicklungsprozess darstellt, der die klassischen Ausmaße des Maschinenbaus übersteigt und daher eine engere Zusammenarbeit zwischen Maschinenbauern und Informatikern erfordert.
\newline
\textbf{Schlüsselwörter:} Industrie 4.0, virtuelle Realität, Menschmodell, Interaktion, Unity Engine

%--------------------------------------------------------------------------------------------------
\begingroup
\let\clearpage\relax
\chapter*{Abstract}\label{sec:Abstract}
The goal of this research project is to investigate the extent to which virtual reality technologies can drive the realization of Industry 4.0. Based on this, the potential of a virtual human image (human model), which can be controlled by the movements of the operator, in combination with a suitable interaction interface for the virtual world, in the context of Industry 4. 0, was investigated.
In order to answer this research question, first of all, after examining the current state of the art in industry, industrial production and virtual reality, requirements were established in a conception phase, which were then taken into account when implementing an application example. The Unity Engine development environment was used for the implementation. Thereupon, the established requirements were validated against the finished implementation.
The resulting application example is able to demonstrate the potential of such a human model in an industrial setting. The human model could, for example, make it possible to inspect and control production plants from any location using a corresponding virtual clone of the plant.
Based on this, it can be clearly stated that virtual reality technologies should be considered in the context of Industry 4.0. Nevertheless, there is still a need for much research and development in this area, since the transformation to Industry 4. 0 represents an interdisciplinary development process that exceeds the classic dimensions of mechanical engineering and therefore requires closer cooperation between mechanical engineers and computer scientists.
\newline
\textbf{Keywords:} Industry 4.0, Virtual Reality, human model, interaction, Unity Engine

\endgroup