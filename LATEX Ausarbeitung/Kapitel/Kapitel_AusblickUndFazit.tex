%--------------------------------------------------------------------------------------------------
\chapter{Ausblick und Fazit}\label{cha:AusblickUndFazit}

Einige der in Kapitel XX dieser Arbeit vorgestellten Potenziale von Industrie 4.0, konnten durch das bei dieser Arbeit entstandene und in Kapitel XX erläuterte Anwendungsbeispiel einer virtuellen Umgebung, in der mittels eines Menschmodells und einer Interaktionsschnittstelle mit beliebigen Objekten interagiert werden kann, verdeutlicht werden.
\newline
Dabei sind die Potenziale zehn und elf besonders hervorzuheben, da mittels der in Kapitel XX vorgestellten Vorgehensweise für die Digitalisierung von Produktionsanlagen und der in Kapitel XX vorgestellten praktischen Umsetzung des Menschmodells, inklusive der Interaktionsschnittstelle, sich die Möglichkeit eröffnet, in Zukunft Produktionsstätten über einen virtuellen Zwilling zu überwachen, mit einzelnen Anlagen zu interagieren und diese somit zu steuern. Aufgrund dessen könnte in Zukunft die Wettbewerbsfähigkeit des Hochlohnstandorts Deutschland verbessert werden, da die Produktionsanlagen über einen virtuellen Zwilling gesteuert und daher an jedem beliebigen Standort auf der Welt stehen könnten.
\newline
Angetrieben, durch den damit verbundenen technologischen Entwicklungsaufwand, eröffnen sich, vor allem für IT-Unternehmen, Chancen am Wandel zu Industrie 4.0 zu profitieren, da der Wandel zu Industrie 4.0 ein von vielen Herausforderungen geprägter und interdisziplinärer Prozess ist und die Ausmaße des klassischen Maschinenbaus übersteigt. Letzteres wird besonders durch die in Kapitel XX vorgestellten Herausforderungen verdeutlicht.

%--------------------------------------------------------------------------------------------------
\section{Ausblick}\label{sec:Ausblick}

Es bleibt die Frage, wie Industrie 4.0 in Zukunft aussehen wird. Durch die gewonnenen Erkenntnisse beim Recherchieren und Verfassen dieser Arbeit, lässt sich eindeutig sagen, dass es keine allgemeine Lösung gibt und geben kann. Wie bereits in Kapitel XX, bei der Vorstellung verschiedener Leitfäden erläutert, stehen Unternehmen vor der schwierigen Herausforderungen, die Industrie 4.0 Potenziale in ihrem Unternehmen zu erkennen und unter Berücksichtigung der in Kapitel XX vorgestellten Herausforderungen umzusetzen. Diese Potenziale können im Zeitalter des Internets der Dinge und Dienstleistungen sowohl im optimieren der Produktionsprozesse oder der Ressourceneffizienz, als auch in der Gestaltung neuer Dienstleistungen (Smart Services) stecken. 
Einziger Anhaltspunkt für Unternehmen sind die bereits erwähnten Leitfäden, wie Beispielsweise die aus Kapitel XX. Dabei ist anzumerken, dass diese mit Hilfe von Werkzeugkästen, Handlungsfeldern, Rahmenbedingungen oder ähnlichem die Unternehmen dabei Unterstützen sollen, einen eigenen Weg für die Umsetzung von Industrie 4.0 zu finden.
\newline\newline
Des Weiteren stellt sich die Frage, inwiefern Virtual Reality (virtuelle Realität) beim Wandel zu Industrie 4.0 eine Rolle spielt. Wie bereits in Kapitel XX dargestellt, befinden wir uns am Anfang einer neuen Phase der Virtual Reality. Dabei findet Virtual Reality neue Einsatzmöglichkeiten in verschiedensten Lebensbereichen, von Bildung bis hin zu Industrie. Als Teil der Disziplin Visual Computing, welche als eine Schlüsselkomponente beim Wandel zu Industrie 4.0 angesehen wird \cite[S.1]{17}, eröffnen sich durch den Einsatz von Virtual Reality viele Potenziale für neue Wertschöpfungsprozesse. Dazu gehört Beispielsweise, die bereits mehrfach erwähnte Möglichkeit einer standortunabhängigen Steuerung von Produktionsanlagen.

%--------------------------------------------------------------------------------------------------3
\section{Zusammenfassung der wichtigsten Ergebnisse}\label{sec:ZusammenfassungErgebnisse}
Zunächst wurde, in Kapitel XX dieser Arbeit, der aktuelle Stand der Technik, insbesondere die Geschichte (Kapitel XX) und die Ausgangslage (Kapitel XX) der industriellen Fertigung und die Geschichte (Kapitel XX) und die Potenziale (Kapitel XX) von Virtual Reality aufgearbeitet.
Daraufhin wurde in Kapitel XX, basierend auf dem Digitalisierungsprozess von Produktionsanlagen (Kapitel XX), der Mensch-Maschine Interaktion (Kapitel XX) und dem FMI Standard (Kapitel XX) ein Konzept geschaffen und Anforderungen definiert (Kapitel XX). Anschließend wurde in Kapitel XX, nach einer Einführung in die verwendete Hardware (Kapitel XX) und Software (Kapitel XX) die Umsetzung des Menschmodells (Kapitel XX) und der Interaktionsschnittstelle (Kapitel XX) erläutert. Schließlich wurde in Kapitel XX die Vorgehensweise beim Einbinden des entstandenen Menschmodells in ein neues Projekt (Kapitel XX) und die Vorgehensweise beim Einfügen neuer Objekte in der Umgebung (Kapitel XX) erläutert, bevor die Anforderungen aus Kapitel XX an das Menschmodell (Kapitel XX) und die Interaktionsschnittstelle (Kapitel XX) validiert wurden.
\newline\newline
Die zentrale Aufgabe dieser Arbeit war es, ein Menschmodell zu schaffen, welches die Bewegungen des Bedieners in der virtuellen Welt verzögerungsfrei Abbildet und dem Bediener ermöglicht mit der virtuellen Umgebung zu interagieren. Dabei mussten insbesondere die letzten zwei, der in Kapitel XX erläuterten Anforderungen Genauigkeit, Echtzeit, Interoperabilität und Modularität gewährleistet werden, um zukünftige Anpassungen am Menschmodell und das problemlose Zusammenfügen des eigentlichen Menschmodells mit der separat entwickelten Interaktionsschnittstelle zu ermöglichen.
\newline
Wie bereits erwähnt, wurde die Interaktionsschnittstelle unabhängig vom Menschmodell, mit Fokus auf die in Kapitel XX gestellten Anforderungen Bidirektionalität, Genauigkeit, Echtzeit, Interoperabilität und Modularität, entwickelt. Sowohl die Anforderungen an das Menschmodell, als auch die Anforderungen an die Interaktionsschnittstelle, konnten durch die Art und Weise der Umsetzung, wie bereits in Kapitel XX erläutert, gewährleistet werden.
\newline
Wegweisend dabei waren die in Kapitel XX gewonnenen Kenntnisse über das Functional-Mockup Interface, da das entstandene Menschmodell, ähnlich wie eine FMU bei einer Co-Simulation, nur ein Bestandteil einer größeren Anwendung ist. So fehlt Beispielsweise, wie in Kapitel XX erläutert, noch eine geeignete Anbindung von Unity an echte Produktionsanlagen (z.B. über eine Cloud) um die Mensch-Maschine Schnittstelle zu vollenden.
\newline\newline
Aufgrund des entstandenen Anwendungsbeispiels offenbart sich die Möglichkeit einer maßgeblichen Wandlung der Industrie, bei der insbesondere die Standortunabhängigkeit von Unternehmen gestärkt wird. Dadurch könnte in Zukunft das zentrale Leitmotiv von Firmen aus dem Produktionsbereich "Designed by us, produced anywhere" [Yübo Wang, M.Sc. M.A.] lauten.
\newline\newline
\newline\newline
1 \cite{1}
2 \cite{2}
3 \cite{3}
4 %\cite{4}
5 \cite{5}
6 \cite{6}
7 \cite{7}
8 \cite{8}
9 \cite{9}
10 \cite{10}
11 \cite{11}
12 \cite{12}
13 \cite{13}
14 \cite{14}
15 \cite{15}
16 \cite{16}
17 \cite{17}
18 %\cite{18}
19 \cite{19}
20 \cite{20}
21 \cite{21}
22 \cite{22}
23 \cite{23}
24 \cite{24}
25 \cite{25}
26 \cite{26}
27 \cite{27}
28 \cite{28}
29 \cite{29}
30 \cite{30}
31 \cite{31}
32 \cite{32}
33 \cite{33}
A1 \cite{A1}
A2 \cite{A2}
A14 \cite{A14}
A15 \cite{A15}
A16 \cite{A16}
A18 \cite{A18}
A26 \cite{A26}
A27 \cite{A27}
A28 \cite{A28}
A29 \cite{A29}
%--------------------------------------------------------------------------------------------------