%--------------------------------------------------------------------------------------------------
\chapter{Ausblick und Zusammenfassung}\label{cha:AusblickUndFazit}

Einige der in Kapitel \ref{sec:PotentialeIndustrie4.0} vorgestellten Potenziale von Industrie 4.0, konnten durch das in Kapitel \ref{cha:Umsetzung} vorgestellte Anwendungsbeispiel einer virtuellen Umgebung, in der mittels eines Menschmodells und einer Interaktionsschnittstelle mit beliebigen Objekten interagiert werden kann, verdeutlicht werden. Dabei sind die Potenziale acht ("Wettbewerbsfähigkeit als Hochlohnstandort" \cite[S.20]{12}), zehn („Simulation und Überwachung der Produktion" \cite{6}) und elf („Chancen für IT-Unternehmen" \cite[S.7]{2}) besonders hervorzuheben.
\newline
Mit Hilfe der in Kapitel \ref{sec:PhysischZumKlon} vorgestellten Vorgehensweise für die Schaffung eines virtuellen Klons einer Produktionsanlagen und mit Hilfe des in Kapitel \ref{cha:Umsetzung} vorgestellten Menschmodells, inklusive der Interaktionsschnittstelle, eröffnet sich die Möglichkeit, in Zukunft Produktionsstätten über einen virtuellen Klon zu begehen, mit einzelnen Produktionsanlagen zu interagieren und diese somit zu steuern. Aufgrund dessen könnte in Zukunft die Wettbewerbsfähigkeit des Hochlohnstandorts Deutschland verbessert werden, da die über einen virtuellen Klon gesteuerten Produktionsanlagen an jedem beliebigen Standort auf der Welt stehen könnten.
Angetrieben durch den damit verbundenen technologischen Entwicklungsaufwand, eröffnen sich vor allem für IT-Unternehmen Chancen am Wandel zur Industrie 4.0 zu profitieren, da der Wandel zur Industrie 4.0 einen von vielen Herausforderungen geprägten und interdisziplinären Prozess darstellt und daher die Ausmaße des klassischen Maschinenbaus übersteigt. Dies wird unter anderem durch die in Kapitel \ref{sec:HerausforderungenUmsetzung} vorgestellten Herausforderungen verdeutlicht. Insbesondere fehlt der zweite Schritt, der in Kapitel \ref{sec:MMInteraktion} erläuterten Mensch-Maschine Interaktion, um die in Unity erschaffene, begehbare und interaktive virtuelle Umgebung mit den realen Produktionsanlagen zu verbinden. Dies war, zum Zeitpunkt des Schreibens dieser Arbeit, das Thema einer weiteren Bachelorarbeit eines Informatik Studenten am Fachgebiet DiK (Datenverarbeitung in der Konstruktion) der Technischen Universität Darmstadt.

%--------------------------------------------------------------------------------------------------
\section{Ausblick}\label{sec:Ausblick}
Es bleibt die Frage, wie Industrie 4.0 in Zukunft aussehen wird. Durch die gewonnenen Erkenntnisse beim Recherchieren und Verfassen dieser Arbeit, lässt sich eindeutig sagen, dass es keine universelle Lösung gibt und geben kann. Wie bereits in Kapitel \ref{sec:LeitfadenUmsetung} angedeutet, stehen Unternehmen vor der schwierigen Herausforderungen, die Industrie 4.0 Potenziale in ihrem Unternehmen zu erkennen und unter Berücksichtigung der in Kapitel \ref{sec:HerausforderungenUmsetzung} vorgestellten Herausforderungen umzusetzen. Diese Potenziale können im Zeitalter des Internets der Dinge und Dienste (IoTS), nicht nur im Optimieren der Produktionsprozesse oder der Ressourceneffizienz, sondern auch in der Gestaltung neuer Dienstleistungen (Smart Services) und vielen weiteren Bereichen liegen (Vgl. Kapitel \ref{sec:PotentialeIndustrie4.0}). 
Ein wichtiger Anhaltspunkt für Unternehmen sind dabei sogenannte Industrie 4.0 Leitfäden (Vgl. Kapitel \ref{sec:LeitfadenUmsetung}). Dabei ist anzumerken, dass diese, mit Hilfe von Werkzeugkästen, Handlungsfeldern, Rahmenbedingungen oder ähnlichem, die Unternehmen lediglich dabei unterstützen sollen, einen eigenen Weg für die Umsetzung von Industrie 4.0 zu finden.
\newline
Des Weiteren stellt sich die Frage, inwiefern Virtual Reality (virtuelle Realität) beim Wandel zur Industrie 4.0 eine Rolle spielt. Wie bereits in Kapitel \ref{sec:VRGeschichte} dargestellt, befinden wir uns am Anfang einer neuen Phase der Virtual Reality. Dabei findet Virtual Reality neue Einsatzmöglichkeiten in verschiedensten Lebensbereichen, von Bildung bis hin zur Industrie. Als Teil der Disziplin Visual Computing, welche als eine Schlüsselkomponente beim Wandel zur Industrie 4.0 angesehen wird \cite[S.1]{17}, eröffnen sich durch den Einsatz von Virtual Reality viele Potenziale für neue Wertschöpfungsprozesse. Dazu gehört z.B. die bereits mehrfach erwähnte Möglichkeit einer standortunabhängigen, virtuellen Steuerung von Produktionsanlagen.
\newline
Außerdem könnte in Zukunft, das bei dieser Arbeit entstandene Anwendungsbeispiel, unter Berücksichtigung des in Kapitel \ref{sec:AusblickFMI} erläuterten Functional-Mockup-Interfaces, erweitert werden und somit als Functional-Mockup Unit bei der Co-Simulation von Produktionsanlagen und -prozessen, als Bestandteil einer beliebigen größeren Anwendung, eingesetzt werden.
%--------------------------------------------------------------------------------------------------3
\section{Zusammenfassung der wichtigsten Ergebnisse}\label{sec:ZusammenfassungErgebnisse}
Zunächst wurde in Kapitel \ref{cha:StandDerTechnik} der aktuelle Stand der Technik, insbesondere im Hinblick auf Industrie 4.0, der Ausgangslage der industriellen Fertigung und der Geschichte von Virtual Reality aufgearbeitet.
Daraufhin wurden in Kapitel \ref{cha:AufbauDesKonzepts}, basierend auf dem Digitalisierungsprozess von Produktionsanlagen und der Mensch-Maschine Interaktion, Anforderungen definiert und der Aufbau des Modells erläutert, bevor in einem Ausblick der FMI Standard und dessen Potenziale vorgestellt wurden.
Anschließend wurde in Kapitel \ref{cha:Umsetzung}, nach einer Einführung in die verwendete Hardware und Software, die Umsetzung des Menschmodells und der Interaktionsschnittstelle mit Hilfe der Unity Engine erläutert. 
Schließlich wurden in Kapitel \ref{cha:ValidierungDesKonzepts} die Vorgehensweisen beim Einbinden des entstandenen Menschmodells in ein neues Projekt und beim Einfügen neuer Objekte in der Umgebung erläutert, bevor die zuvor gestellten Anforderungen an das Menschmodell und die Interaktionsschnittstelle validiert wurden.
\newline\newline
Die zentrale Aufgabe dieser Arbeit war es, ein Menschmodell zu schaffen, welches die Bewegungen des Bedieners in der virtuellen Welt nahezu verzögerungsfrei abbildet und dem Bediener ermöglicht, mit der virtuellen Umgebung zu interagieren. Dabei mussten insbesondere die letzten zwei, der in Kapitel \ref{sec:AnforderungenKonzept} erläuterten Anforderungen Genauigkeit, Echtzeit, Interoperabilität und Modularität, gewährleistet werden, um zukünftige Anpassungen am Menschmodell und das problemlose Zusammenfügen des eigentlichen Menschmodells mit der separat entwickelten Interaktionsschnittstelle zu ermöglichen.
Aufgrund dessen, wurde die Interaktionsschnittstelle unabhängig vom Menschmodell, mit Fokus auf die in Kapitel \ref{sec:AnforderungenKonzept} gestellten Anforderungen Bidirektionalität, Genauigkeit, Echtzeit, Interoperabilität und Modularität, entwickelt. Sowohl die Anforderungen an das Menschmodell, als auch die Anforderungen an die Interaktionsschnittstelle, konnten durch die Art und Weise der Umsetzung, gewährleistet werden (Vgl. \ref{sec:ValidMensch} und \ref{sec:ValidInteraktion}).
Die Vollendung der Mensch-Maschine Schnittstelle, also das Verbinden der virtuellen Umgebung in Unity mit echten Produktionsanlagen, durch das Zusammenfügen mit der bereits angesprochenen Arbeit eines anderen Studenten am DiK, sollte, dank der Einhaltung der Anforderungen an die Modularität und Interoperabilität, mit geringen Aufwand möglich sein.
\newline\newline
Aufgrund des entstandenen Anwendungsbeispiels, vor allem in Kombination mit dem zweiten Schritt der Mensch-Maschine Interaktion, offenbart sich die Möglichkeit einer maßgeblichen Wandlung der Industrie, bei der insbesondere die Standortunabhängigkeit von Unternehmen gestärkt wird. Dadurch könnte in Zukunft das zentrale Leitmotiv von Firmen aus dem Produktionsbereich "Designed by us, produced anywhere" (Yübo Wang, M.Sc. M.A.) lauten.
%1 \cite{1} 2 \cite{2} 3 \cite{3} 4 \cite{4} 5 \cite{5} 6 \cite{6} 7 \cite{7} 8 \cite{8}
%9 \cite{9} 10 \cite{10} 11 \cite{11} 12 \cite{12} 13 \cite{13} 14 \cite{14} 15 \cite{15}
%16 \cite{16} 17 \cite{17} %18 \cite{18} 
%19 \cite{19} 20 \cite{20} 21 \cite{21} 22 \cite{22}
%23 \cite{23} 24 \cite{24} 25 \cite{25} 26 \cite{26} 27 \cite{27} 28 \cite{28} 29 \cite{29}
%30 \cite{30} 31 \cite{31} 32 \cite{32} 33 \cite{33} 34 \cite{34} A1 \cite{A1} A2 \cite{A2} 
%A14 \cite{A14} A15 \cite{A15} A16 \cite{A16} A18 \cite{A18} A26 \cite{A26} A27 \cite{A27} 
%A28 \cite{A28} A29 \cite{A29}
%--------------------------------------------------------------------------------------------------